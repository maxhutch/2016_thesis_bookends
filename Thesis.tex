%% ----------------------------------------------------------------
%% Thesis.tex -- MAIN FILE (the one that you compile with LaTeX)
%% ---------------------------------------------------------------- 

% Set up the document
\documentclass[a4paper, 11pt, oneside]{Thesis}  % Use the "Thesis" style, based on the ECS Thesis style by Steve Gunn
\graphicspath{Figures/}  % Location of the graphics files (set up for graphics to be in PDF format)

% Include any extra LaTeX packages required
\usepackage[square, numbers, comma, sort&compress]{natbib}  % Use the "Natbib" style for the references in the Bibliography
\usepackage{verbatim}  % Needed for the "comment" environment to make LaTeX comments
\usepackage{vector}  % Allows "\bvec{}" and "\buvec{}" for "blackboard" style bold vectors in maths
\usepackage{mysty}
\hypersetup{urlcolor=blue, colorlinks=true}  % Colours hyperlinks in blue, but this can be distracting if there are many links.

%% ----------------------------------------------------------------
\begin{document}
\frontmatter      % Begin Roman style (i, ii, iii, iv...) page numbering

% Set up the Title Page
\title  {Data driven modeling of the low-Atwood single-mode Rayleigh-Taylor instability}
\authors  {\texorpdfstring
            {\href{maxhutch@uchicago.edu}{Maxwell Hutchinson}}
            {Maxwell Hutchinson}
            }
\addresses  {\groupname\\\deptname\\\univname}  % Do not change this here, instead these must be set in the "Thesis.cls" file, please look through it instead
\date       {\today}
\subject    {}
\keywords   {}

\maketitle
%% ----------------------------------------------------------------

\setstretch{1.3}  % It is better to have smaller font and larger line spacing than the other way round

% Define the page headers using the FancyHdr package and set up for one-sided printing
\fancyhead{}  % Clears all page headers and footers
\rhead{\thepage}  % Sets the right side header to show the page number
\lhead{}  % Clears the left side page header

\pagestyle{fancy}  % Finally, use the "fancy" page style to implement the FancyHdr headers

%% ----------------------------------------------------------------
% Declaration Page required for the Thesis, your institution may give you a different text to place here
\Declaration{

\addtocontents{toc}{\vspace{1em}}  % Add a gap in the Contents, for aesthetics

I, Maxwell Hutchinson, declare that this thesis titled, `Data driven modeling of the low-Atwood single-mode Rayleigh-Taylor instability' and the work presented in it are my own. I confirm that:

\begin{itemize} 
\item[\tiny{$\blacksquare$}] This work was done wholly or mainly while in candidature for a research degree at this University.
 
\item[\tiny{$\blacksquare$}] Where any part of this thesis has previously been submitted for a degree or any other qualification at this University or any other institution, this has been clearly stated.
 
\item[\tiny{$\blacksquare$}] Where I have consulted the published work of others, this is always clearly attributed.
 
\item[\tiny{$\blacksquare$}] Where I have quoted from the work of others, the source is always given. With the exception of such quotations, this thesis is entirely my own work.
 
\item[\tiny{$\blacksquare$}] I have acknowledged all main sources of help.
 
\item[\tiny{$\blacksquare$}] Where the thesis is based on work done by myself jointly with others, I have made clear exactly what was done by others and what I have contributed myself.
\\
\end{itemize}
 
 
Signed:\\
\rule[1em]{25em}{0.5pt}  % This prints a line for the signature
 
Date:\\
\rule[1em]{25em}{0.5pt}  % This prints a line to write the date
}
\clearpage  % Declaration ended, now start a new page

%% ----------------------------------------------------------------
% The "Funny Quote Page"
\pagestyle{empty}  % No headers or footers for the following pages

\null\vfill
% Now comes the "Funny Quote", written in italics
\textit{``Bubbler morghulis -- all bubbles must die''}

\begin{flushright}
No one
\end{flushright}

\vfill\vfill\vfill\vfill\vfill\vfill\null
\clearpage  % Funny Quote page ended, start a new page
%% ----------------------------------------------------------------

% The Abstract Page
\addtotoc{Abstract}  % Add the "Abstract" page entry to the Contents
\abstract{
\addtocontents{toc}{\vspace{1em}}  % Add a gap in the Contents, for aesthetics
The Rayleigh-Taylor instability is one of the most common and well studied phenomena in fluid dynamics. 
%, appearing and impacting such diverse flows as the doubly diffusive salt finger, nuclear flames in type 1a supernovae, and ablator mixing in inertial confinement fusion.
Despite research dating to 
%its characterization by Lord Rayleigh in 
the late 19th century, the non-linear dynamics 
%of bubbles originating 
of the interfacial instability are still not fully understood, particularly in the case when the two fluids have nearly the same density.
It was recently demonstrated in this, the low-Atwood regime, that the idealized single-mode problem departs from established potential flow models in the form of a re-acceleration beyond the predicted terminal interface velocity.
This thesis is an attempt to model that re-acceleration and, more broadly, the late time dynamics of the single-mode low-Atwood Rayleigh-Taylor instability.

The approach taken here is based on buoyancy-drag models, which express a force balance between buoyancy and parasitic drag.
The dynamical buoyancy-drag model is supplemented with a mixing model that dilutes the buoyant force over time.
These models are written deliberately generally, with 8 unique coefficients.
Three of these coefficients are solved for by equating the early time behavior with that of well established linear theories.
The remaining 5 coefficients are estimated by relating them to drag coefficients, friction factors, and geometric ratios in the interface shape.

To evaluate the model and compute the 5 unknown coefficients more precisely, a set of direct numerical simulations are performed over the relevant parameter space.
These simulations are first validated against experimental data.
Then they are shown to converge and their resolutions are chosen such as to minimize computational cost given the accuracy scale of the model.
The 5 coefficients are fit to the resulting data set, and the model achieves better than 2\% error in the bubble height and 4\% error in the volume of mixed fluid.
Three coefficients are nominally independent of the parameterization of the problem, while two are shown to vary with the Rayleigh number and the diffusivity.

%The approach taken in this thesis is a starting point for data-driven modeling of model fluid dynamics problems.
%The model could be augmented with a vorticity term, possibly capturing re-acceleration.
%Many of the simulated cases are incomplete in that the Rayleigh-Taylor front reached the top and bottom boundaries of the simulated domain.
%The computational cost is shown to go with the Rayleigh number to the 6th power, making the compleition of these cases expensive.
}

\clearpage  % Abstract ended, start a new page
%% ----------------------------------------------------------------

\setstretch{1.3}  % Reset the line-spacing to 1.3 for body text (if it has changed)

% The Acknowledgements page, for thanking everyone
\acknowledgements{
\addtocontents{toc}{\vspace{1em}}  % Add a gap in the Contents, for aesthetics

I would like to thank Robert Rosner for letting me chart my course through this thesis and my doctoral research more broadly.
I thank Aleksandr Obabko and Paul Fischer for supporting Nek5000 and working with me through the numerics.
I also thank Oana Marin and Michel Schanen for their excitement and Elia Merzari and Ron Rahaman for their support.
I thank Alexander Heinecke and Scott Parker for their help tuning the code to run on supercomputers.
I thank Elizabeth Hicks for helpful comments and advice.

For computer time, this research partially used the resources of the
Supercomputing Laboratory at King Abdullah University of Science \& Technology
 (KAUST) in Thuwal, Saudi Arabia.
This research used resources of the Argonne Leadership Computing Facility, which
is a DOE Office of Science User Facility supported under Contract DE-AC02-06CH11357.

}
\clearpage  % End of the Acknowledgements
%% ----------------------------------------------------------------

\pagestyle{fancy}  %The page style headers have been "empty" all this time, now use the "fancy" headers as defined before to bring them back


%% ----------------------------------------------------------------
\lhead{\emph{Contents}}  % Set the left side page header to "Contents"
\tableofcontents  % Write out the Table of Contents

\begin{comment}
%% ----------------------------------------------------------------
\lhead{\emph{List of Figures}}  % Set the left side page header to "List if Figures"
\listoffigures  % Write out the List of Figures

%% ----------------------------------------------------------------
\lhead{\emph{List of Tables}}  % Set the left side page header to "List of Tables"
\listoftables  % Write out the List of Tables

%% ----------------------------------------------------------------
\setstretch{1.5}  % Set the line spacing to 1.5, this makes the following tables easier to read
\clearpage  % Start a new page
\lhead{\emph{Abbreviations}}  % Set the left side page header to "Abbreviations"
\listofsymbols{ll}  % Include a list of Abbreviations (a table of two columns)
{
% \textbf{Acronym} & \textbf{W}hat (it) \textbf{S}tands \textbf{F}or \\
\textbf{LAH} & \textbf{L}ist \textbf{A}bbreviations \textbf{H}ere \\

}
\end{comment}

%% ----------------------------------------------------------------
%\clearpage  % Start a new page
%\lhead{\emph{Physical Constants}}  % Set the left side page header to "Physical Constants"
%\listofconstants{lrcl}  % Include a list of Physical Constants (a four column table)
%{
% Constant Name & Symbol & = & Constant Value (with units) \\
%Speed of Light & $c$ & $=$ & $2.997\ 924\ 58\times10^{8}\ \mbox{ms}^{-\mbox{s}}$ (exact)\\
%
%}

%% ----------------------------------------------------------------
\clearpage  %Start a new page
\lhead{\emph{Symbols}}  % Set the left side page header to "Symbols"
\listofnomenclature{lll}  % Include a list of Symbols (a three column table)
{
% symbol & name & unit \\
D & Mass diffusivity & m$^2$ / s \\
g & Frame acceleration &  m / s$^2$ \\
P & Pressure & kg / (m s$^2$)\\
$\lambda$ & Single-mode wavelength & m \\
$\nu$ & Kinematic viscosity & m$^2$ / s \\
$\rho_h$ & Higher density & kg / m$^3$ \\
$\rho_l$ & Lower density & kg / m$^3$  \\
\hline 
A & Atwood number & $\frac{\rho_h - \rho_l}{\rho_h + \rho_l}$ \\
Gr & Grashof number & $ \frac{A g \lambda^3}{\nu^2}$ \\
Re$_p$ & Perturbation Reynolds number & $\sqrt{\text{Gr}}$ \\
Ra & Rayleigh number & $ \frac{A g \lambda^3}{\nu D}$ \\
Sc & Schmidt number & $\nu$ / D  \\
}
%% ----------------------------------------------------------------
% End of the pre-able, contents and lists of things
% Begin the Dedication page

\setstretch{1.3}  % Return the line spacing back to 1.3

\pagestyle{empty}  % Page style needs to be empty for this page
\dedicatory{Dedicated to Tracey Ziev}

\addtocontents{toc}{\vspace{2em}}  % Add a gap in the Contents, for aesthetics


%% ----------------------------------------------------------------
\mainmatter	  % Begin normal, numeric (1,2,3...) page numbering
\pagestyle{fancy}  % Return the page headers back to the "fancy" style

% Include the chapters of the thesis, as separate files
% Just uncomment the lines as you write the chapters

\lhead{}  % Set the left side page header to "Symbols"
\chapter{Introduction}

\section{Formal definition}
The Rayleigh-Taylor instability occurs when the pressure and density gradients are in opposition:
\begin{equation}
(\nabla P)(\nabla \rho) < 0
\end{equation}
The canonical example of the Rayleigh-Taylor instability is a heavy fluid superposed over a lighter one in a gravitational field.
The standard terminology is based on this case.

Consider a horizontal planar interface at $z=0$ between a fluids of densities $\rho_h > \rho_l$ with the denser fluid at $z > 0$ and the lighter at $z < 0$, and a gravitational acceleration $-g \hat{z}$.
In equilibrium, the pressure must balance the gravitational force:
\begin{equation}
P(x,y,z) = \begin{cases}- \rho_h g z + C& \text{ if } z > 0 \\
                        - \rho_l g z + C& \text{ else}
           \end{cases}
\end{equation}
A small perturbation is introduced, moving some heavy fluid below the interface and some light fluid above it.
The forcing on the heavier fluid below the interface is:
\begin{equation}\elabel{force1}
\sum F = - \nabla P + F_g = \rho_l g  - \rho_h g < 0
\end{equation}
so the heavier fluid is forced downwards through the lighter fluid.
Conversely, the forcing on the lighter fluid above the interface is:
\begin{equation} \elabel{force2}
\sum F = - \nabla P + F_g = \rho_l g  - \rho_h g > 0
\end{equation}
so the lighter fluid is forced upwards through the heavier fluid.
Perturbations in the interface grow, so the configuration is unstable.

\section{Instances and motivation}
The Rayleigh-Taylor instability is present in both natural and constructed systems at many scales.
This section describes a few such systems of particular importance.

\paragraph{Supernovae}
One of the earliest motivations for the study of the Rayleigh-Taylor instability comes from the stability 

\paragraph{Salt fingers}
Salt fingers are an instance of the Rayleigh-Taylor instability set up by a difference in the diffusivity of two mass carriers, in this case salinity and temperature.

\paragraph{ICF}
Inertial confinement fusion (ICF) is a fusion technique that confines hot dense plasma temporarily via an implosion, rather than magnetic field lines.
In ICF, the cryogenic hydrogen is coated with a plastic ablator forming very small hollow spheres, or microcapsules.
The microcapsules are through a cylindrical hohlraum.
When the capsule is at the center of the hohlraum, the hohlraum is illuminated with a high energy burst of laser light and radiates x-rays.
The x-ray radiation is absorbed by the plastic ablator causing it to rapidly expand and blow off the microcapsule, creating an implosion in the hydrogen fuel.

The implosion isn't perfectly uniform; the x-ray radition provides non-uniform acceleration and there are asymmetries in the microcapsue.
These perturbations are Rayleigh-Taylor unstable: the dense plastic ablator is being accelerated through lighter hydrogen fuel.
The carbon in the ablator is much better at radiating energy than the hydrogen, so as the RTI mixes the plastic into the fuel the fuel cools, preventing ignition.

\paragraph{Reactors}


\section{Terminology}

\paragraph{Atwood number}
The Atwood number characterizes the density contrast:
\begin{equation} \elabel{atwood}
A = \frac{\rho_1 - \rho_2}{\rho_1 + \rho_2} \in (-1,1)
\end{equation}
where $\rho_1 > \rho_2$ corresponds to positive Atwood number.

There are three distinct regimes for the Atwood number: high, low, and moderate.
At high Atwood number, e.g. air and water, the internal dynamics of the light fluid are decoupled from the heavy fluid.
The transfer of momentum through the interface into the dense fluid can be neglected.
In this regime, potential flow models are reasonable.

At low Atwood number, e.g. salt water and fresh water, the governing equations can be linearized about the Atwood number, leading to the Boussinesq approximation.
The behavior of the bubbles and spikes are identical.

At moderate Atwood number, e.g. oil and water, not only is enough vorticity is generated to invalidate potential flow models but also the density difference breaks bubble-spike symmetry.

\paragraph{Boussinesq approximation}
If the density contrast, $\Delta \rho$ is small compared to the average density $\bar{\rho}$, then 
the governing equations can be linearized with respect to density.
This is known as the Boussinesq approximation and has the primary effect of neglecting differences in the inertia of the two fluids.

\paragraph{Constant property}
In the spirit of the Boussinesq approximation, which neglects differences in the inertial of the two fluids, we furhter assume the two fluids share all other material properties, such as viscosity.

\paragraph{Miscible interface}
In most cases where the Boussinesq and constant property approximations are valid, the interface between the two fluids is miscible.
For example, low-concetration solutions in a common solvent have interfaces governed by a diffusion coefficient, $D$, and small temperature graidents are governed by a thermal diffusivity $\alpha$.
In the limit where $D$ or $\alpha$ goes to zero, the interface is immiscible but there is no surface tension.

\paragraph{Single-mode}

\paragraph{Multi-mode}

\paragraph{Governing equations}
Given a miscible interface between two Boussinesq,  constant property fluids, the governing equations can be recaste in terms of an active scalar:
\begin{align}
\der{u}{t} + u \cdot \nabla u &= \nu \nabla^2 u - \nabla P - A g \phi \hat{z}\\
\der{\phi}{t} + u \cdot \nabla \phi &= D \nabla^2 \phi \\
\nabla \cdot u  &= 0
\end{align}
These equations have three governing parameters; $Ag$, $\nu$, and $D$; which admit the construction of two dimensionless numbers:
\begin{equation}
\text{Grashof} = \frac{Ag L^3}{\nu^2}
\end{equation}
\begin{equation}
\text{Schmidt} = \frac{\nu}{D}
\end{equation}
If the initial and boundary conditions can be characterized by a single length scale, then these two parameters uniquely identify the system.

\paragraph{Bubble height / spike depth / mixing width}

Experiment and theory focus on the observation and explanation of a set of observables that are much smaller than the full degrees of freedom of the system.
The selection of these observables is informed by practical aplication and methodological analogy.


The bubble height refers to the distance beyond the initial interface that the bubble front has traveled, as a function of time.
The standard experimental definition projects the density onto a line normal to the initial interface and defines the front interface as the point at which the density is at its 99th or 95th percentile.
More formally:
\begin{equation}
H[\epsilon] = \sup \left\{z : \int dx dy \rho(x,y,z) < (1-\epsilon) \int dx dy \rho(x,y,\infty) \right\}
\end{equation}



\begin{comment}
\section{Stages}

\subsection{Linear growth} 
When the amplitude of the surface perturbation is small compared to its wave-length, the perturbation grows exponentially.
The growth rate is seen to depend on the forcing, wave-length, viscosity, diffusivity, and interface thickness.
This stage can be treated with linear and weakly-nonlinear theories, as discussed further in \sref{linear}.


\subsection{Single-mode stagnation}
As the growing Rayleigh-Taylor modes saturate, the acceleration of the bubble tip decreases.
The stagnation stage of the single-mode Rayleigh-Taylor instability is defined as a stage of the flow that exhibits nearly constant bubble velocity.
It had been believed that the first such stage was terminal, i.e. steady state.
It has recently been shown, via both simulation~\cite{Ramaprabhu2006} and experiment~\cite{Wilkinson2007}, that at low Atwood number and high Reynolds number the flow continues to develop into re-acceleration stage.
For this reason, the nearly constant velocity is termed the \textit{stagnation velocity}.
The primary observable of a stagnation stage is the stagnation velocity, but there is secondary interest in the velocity profile through the transition from the weakly non-linear to stagnation stages and in the bubble height at which stagnation velocity is reached.

\subsection{Single-mode late time}
For low Atwood and high Reynolds numbers, the stagnation stage is followed by a reacceleration stage around unity aspect ratio.
Reacceleration has been seen in both numerical simulations \cite{Ramaprabhu2006, Ramaprabhu2012, Wei2012} and experiments{Wilkinson2007}.
Following the immediate reacceleration, there is disagreement as to whether or not the single mode problem ultimatly reaches a terminal stage.
Ramaprabhu \etal suggest that the re-acceleration is transient, with the bubble velocity ultimately returning to a potential-flow-like value.
Wei and Livescu, on the other hand, suggest the terminal stage to be one of `chaotic development` with quadratic growth dynamics.
Unfortunately, the late-time stages have yet to be accessed experimentally.

\subsection{Multi-mode}
Under natural multi-mode initial conditions, the linear growth modes couple as the aspect ratio increases, forming bubble and spike structures.
The bubbles and spikes interact with one another, competetively and constructively, leading to successively larger structures.
It is generally observed that the multi-mode aspect ratio, $D_b / h$, remains nearly constant.
\end{comment}
 % Introduction

\chapter{Background}

The study of the Rayleigh-Taylor instability is primarily interested in evolution of the interface, that is the rate of penetration of the light fluid into the dense one and vice versa.
While the volumetric mixing rate is relevant in some contexts, most flows have relatively low diffusivities, i.e. high Prandtl and Schmidt number, so mixing is dominated by transport rather than diffusion.

In this section, we review approaches to modeling the evolution of the interface and experimental efforts to validate those models.


\section{Linear and weakly non-linear models} %\slabel{linear}

The earliest models of the Rayleigh-Taylor instability were based on a linearization of the governing equations about the interface.
Recently, with the aid of computational algebra, it has become possible to retain higher order terms in the expansion, demonstrating the coupling between different modes.
However, even high order expansions fail as the interface loses analyticity.

\subsection{Lord Rayleigh's linear model}

Lord Rayleigh considered a sinusoidal perturbation of an incompressible, inviscid, immiscible, quiescent stratified interface~\cite{Rayleigh1883}.
When the amplitude is small compared to the wavelength, the continuity and momentum equations can be linearized.
\begin{align}
\left(\bar\rho + \tilde\rho\right) \left[\der{u}{t} + u \nabla u \right] &= - \nabla{\tilde{P}} + g(\bar\rho + \tilde\rho)\\
\nabla \cdot u &= 0 \\
\end{align}
where $u, \tilde\rho, \tilde{P}$ are small and their products neglected.
The solution is an exponential with a growth rate:
\begin{equation} \elabel{simple_growth}
  \gamma^2 = A g k 
\end{equation}
where $g$ is the acceleration experienced by the fluid, and
$k = 2 \pi / \lambda$ is the wave-number of the perturbation.
Positive Atwood numbers correspond to unstable density stratifications, which grow exponentially.
Negative Atwood numbers correspond to stable density stratifications, which oscillate.
For a derivation of \eref{simple_growth}, see \aref{LST}.

\subsection{Viscous and diffusive linear models}
Chandrasekhar~\cite{Chandrasekhar1955} and Hide~\cite{Hide1955} generalized the linear theory to viscous fluids by including an isotropic incompressible Newtonian shear stress.
Chandrasekhar worked out the constant property, $\mu_1 = \mu_2$ case and Hide included an approximate combination of distinct viscosities.
Here, we are concerned with the simpler constant propety case:
\begin{equation} \elabel{visc_growth}
  \gamma = \sqrt{A g k + \nu^2 k^4} - \nu k^2
\end{equation} 
where 
$\nu$ is the kinematic viscosity.

LeLevier et al.~\cite{LeLevier1955} generalized the linear theory to continuous density gradients, specifically exponentially smoothed profiles for the form $\bar\rho \pm e^{\mp K z} \delta\rho$.
\begin{equation}
\gamma = \sqrt{\frac{A g k K}{k + K}}
\end{equation}

Duff \etal~\cite{Duff1962} generalized the linear theory to miscible interfaces and incorporated Chandrasekhar and Hide's viscous theories, producing an combined expression for the growth rate:
\begin{equation} \elabel{duff_simple}
\gamma = \sqrt{\frac{A g k}{\psi(A,k\delta)} + \nu^2 k^4} - (\nu + D) k^2
\end{equation}
where 
$\delta$ is the instantaneous interface thickness,
$D$ is the diffusivity,
and $\psi$ is a function of the Atwood number and the product of the wavenumber and the interface thickness.
For $k \delta \lesssim 1$ and $A << 1$, $\psi \approx 1 + k \delta / \sqrt{\pi}$.

In the constant propety case, $\delta = 2 \sqrt{D t}$, introducing a time-dependence on the linear stability:
\begin{equation} \elabel{duff_growth}
\gamma = \sqrt{\frac{A g k}{1 + \frac{2 k}{\sqrt{\pi}}\sqrt{D t} } + \nu^2 k^4} - (\nu + D) k^2
\end{equation}

\subsection{Weakly nonlinear expansions}
Jacobs and Catton provide a third order weakly non-linear theory for the inviscid unit Atwood Rayleigh-Taylor instability~\cite{Jacobs1988}.
Their weakly non-linear theory is primarily used to compare linear growth rates across a variety of perturbation symmetries in 3D.
In particular, hexagonal and axi-symmetric perturbations are found to grow faster than rectangular perturbations.

Berning and Rubenchik extend the theory to arbitrary Atwood immiscible flows at higher order, but take only the third order~\cite{Berning1998}.
They perform a similar geometric comparison to Jacobs and Catton, but also use the harmonic couplings to characterize linear saturation.

The perturbation expansion has been taken to at least the 10th order by Liu \etal~\cite{Wang2010}.
However, there is limited progress to be made with such expansions, as singularities with branching point structures develop at moderate bubble displacements~\cite{Berning1998}.
Put another way, the interface and velocity potentials are not analytic in the span-wise position, e.g. when the interface rolls up.

\section{Potential flow models}

The next class of models to be applied to the Rayleigh-Taylor instability are potential flow models.
These models assume that little vorticity is generated and that it is confined to the interface, which is true at high Atwood numbers.

\subsection{Layzer's unit Atwood model}

One of the first such models is due to Layzer~\cite{Layzer1955}.
Layzer's model is of an bubble with $\rho = 0$ rising in a fluid of density $\rho = 1$ ($A = 1$).
The bubble and fluid are assumed to be incompressible and inviscid.
The flow begins at rest, so there is no initial vorticity.
Layzer claims the flow will therefore continue to be irrotational, because the viscous generation term of the vorticity equation is zeroed for inviscid, incompressible flows.

Since the flow is inviscid and irrotational, Layzer uses the potential flow technique, writing the velocity as the gradient of a scalar potential:
\begin{equation}
v = \nabla \phi
\end{equation}
where 
$v$ is the velocity and 
$\phi$ is the scalar potential.
Incompressibility zeroes the Laplacian of the potential:
\begin{equation}
\nabla^2 \phi = 0
\end{equation}
A Bernoulli equation is used model the interface:
\begin{equation} \label{eqn:LayzerBernoulli}
\der{\phi}{t}(\eta(r,t), r, t) - \frac{1}{2} \left(\left(\pder{\phi}{z}\right)^2(\eta(r,t), r, t) +\left(\pder{\phi}{r}\right)^2(\eta(r,t), r, t)\right) - g \eta(r,t) = f(t)
\end{equation}
where 
$\eta(r,t)$ is the height of the interface,
$g$ is the gravitational acceleration, and 
$f(t)$ is an arbitrary function of time but not space.
The flow is axially symmetric with a vanishing radial component at transverse walls and vanishing vertical component far away from the bubble:
\begin{equation}
\pder{\phi}{r}(z,R,t) = 0 \qquad \pder{\phi}{z}(\pm \infty, r, t) = 0
\end{equation}
Finally, the fluid advects the interface:
\begin{equation}
\der{\eta}{t}(r,t) = \pder{\phi}{z}(\eta(r,t), r, t) - \pder{\phi}{r}(\eta(r,t), r, t) \der{\eta}{r}(r,t)
\end{equation}

The details of the derivation can be found in \aref{Layzer} and the results are summarized here.
The stagnation velocity in two and three dimensions, assuming axial symmetry in the latter case, are:
\begin{equation}
V_{2d} = \frac{1}{\sqrt{3}} \sqrt{\frac{g R}{\pi}}  \qquad V_{cyl} = \sqrt{\frac{g R}{\beta_1}} 
\end{equation}
where $\beta_1$ is the first root of the first order Bessel function of the first kind: $J_{1}(\beta_1) = 0$.

\subsection{Goncharov's high Atwood model}

Goncharaov extends the Layzer model to include two fluids of arbitrary density difference and makes a different choice of simplifying approximation for the Bernoulli equation~\cite{Goncharaov2002}.
The consideration of a second fluid with non-zero density turns the Bernoulli equation \eref{LayzerBernoulli} into a difference:
\begin{equation} \label{eqn:GonBernoulli}
\begin{aligned}
&\der{\phi}{t}(\eta(r,t), r, t) - \frac{1}{2} \left(\left(\pder{\phi}{z}\right)^2(\eta(r,t), r, t) +\left(\pder{\phi}{r}\right)^2(\eta(r,t), r, t)\right) - g \eta(r,t) \\
&= f(t) \\
\end{aligned}
\end{equation}
The Goncharaov model keeps the free-slip boundary condition between the two fluids, which is exact only for $A = 1$ and a reasonable approximation for $\rho_1 / \rho_2 >> 1$.
In this respect, Goncharaov's should be reasonable for high-Atwood, high-Reynolds number flows.

The assumption that the flow is irrotational applies only at high Atwood number.
At moderate and low Atwood numbers, secondary Kelvin-Helmholtz (KH) instabilities develop at the vertical sheer layer separating the two fluids~\cite{Ramaprabhu2006,Wilkinson2007}.
The effects of the vorticity in the lighter fluid were addressed theoretically by Banerjee et al~\cite{Banerjee2011} and numerically by Ramaprabhu et al.~\cite{Ramaprabhu2012}.

\subsection{Abarzhi and Sohn's models}

\subsection{Departure from potential flow}

\subsection{Banerjee's rotational model}

\cite{Banerjee2011}

\section{Buoyancy-drag models}

Buoyancy-drag models were developed concurrently with potential flow models, in part to provide a physical interpretation for their results.
They balance buoyant and parasitic forces related to the geometry of a model bubble.
Historically, buoyancy-drag models have had only 1 or 2 adjustable parameters, so they are evaluated more on their ability to reproduce specific features of the flow, e.g. the terminal velocity, rather than the full time-history.
Here, we focus on models applicable to single-mode non-interacting bubbles.

\subsection{Bubble model of Davies and Taylor}

Early experiments on the Rayleigh-Taylor instability by Davies and Taylor~\cite{Davies1950a} were performed by measuring the dynamics of large bubbles of gas rising through a dense liquid.
In their analysis, they relate the terminal velocity of the bubble to a drag coefficient, implicitly defining a bouyancy-drag model of the form:
\begin{equation} \elabel{dtbd}
\dot{v} \rho \mathcal{V} = \rho g \mathcal{V} - C_D \pi D^2 \frac{1}{2} \rho v^2,
\end{equation}
where $v$ is the gas bubble velocity,
$\rho$ is the density of the liquid,
$g$ is the gravitational acceleration,
$\mathcal{V}$ is the bubble volume,
$C_D$ is a drag coefficient, and
$D$ is the bubble diameter.

%also \cite{Alon1995}

\subsection{Tube model of Dimonte and Schneider}

Dimonte and Schneider develop a buoyancy-drag model for tube-shaped bubbles~\cite{Dimonte1996,Dimonte2000a} based on Davies and Taylor's model, \eref{dtbd}.
They let the ratio of the area to the volume go with the inverse bubble height, $\mathcal{A} / \mathcal{V} \sim 1/h$.
They also add a rescaling of the buoyant term by $\beta$, attributed to Youngs:
\begin{equation}
\dot{v_b}  = \beta A g - C_d \frac{v_b^2}{h_b}, 
\end{equation}
where $v_b$ is the bubble velocity,
$\beta < 1$ accounts for the relatively smaller buoyant portion of the bubble due to entrainment,
$C_d$ is a drag coefficient, and
$h_b$ is the bubble height.
$\beta$ and $C_d$ depend on the Atwood number, but Dimonte proposes $\beta = 1/2$ and $C_d = 2$ for $A << 1$~\cite{Dimonte2000}.
However, the model is stated to apply to self-similar bubble fronts.

\subsection{Self-similar model of Oron}

A simple model of the stagnation stage is buoyancy-drag~\cite{Oron2001}:
\begin{equation} \elabel{buoyancy_drag}
(\rho_1 + C_a \rho_2) \mathcal{V} \ddot{h} = (\rho_2 - \rho_1) \mathcal{V} g - C_d \dot{h}^2 \rho_2 \mathcal{A}
\end{equation}
where $\rho_2 > \rho_1$ are the densities of the two fluids, 
$C_a$ is an added mass coefficient,
$\mathcal{V}$ is the volume of the bubble, 
$h$ is the height of the bubble,
$g$ is the gravitational acceleration,
$C_d$ is a drag-like coefficient, and
$\mathcal{A}$ is the bubble surface area.
There is ambiguity as to the value of un-subscripted $\rho$, but in the low Atwood number limit $\rho_1 \approx \rho_2 = \rho$ removes it:
\begin{equation}
\ddot{h} = A g - \frac{C}{2} \dot{h}^2 \frac{\mathcal{A}}{\mathcal{V}}
\end{equation}

There are two length scales in the single mode instability: $\lambda$ and $h$.
If the bubble is self-similar or unsupported, as is generally the case in the literature, then $\mathcal{A}/\mathcal{V} \sim 1/\lambda$.
Rolling that constant into $C$ yields:
\begin{equation} \elabel{bouancy_drag}
\ddot{h} = A g - C \frac{\dot{h}^2}{2 \lambda}  
\end{equation}
The stagnation velocity is therefore:
\begin{equation}
v_s = \sqrt{\frac{2 A g \lambda}{C}}
\end{equation}
This motivates the definition of the Froude-type number:
\begin{equation}
\text{Fr} = \frac{v}{\sqrt{\frac{A g \lambda}{1+A}}} \xrightarrow{A << 1} \frac{v}{\sqrt{A g \lambda}}
\end{equation}
which relates to the drag-like coefficient as:
\begin{equation}
\text{Fr} = \sqrt{\frac{2}{C}}
\end{equation}


\section{Problems with single mode buoyancy-drag models}

It has recently been shown, via both simulation~\cite{Ramaprabhu2006} and experiment~\cite{Wilkinson2007}, that at low Atwood number and high Reynolds number the flow continues to develop beyond the terminal velocity predicted by potential flow and buoyancy-drag into re-acceleration stage.
For this reason, the nearly constant velocity is termed the \textit{stagnation velocity}.

The desire to describe this re-acceleration process, quantitatively, is the motivation for this thesis.
However, I will first give two qualitative explanations for while re-acceleration should be expected: one based on the pressure balance and another by identifying a historical inconsistency in the development of buoyancy-drag models.

\subsection{Pressure in the single-mode RTI}

If there is a terminal velocity regime, can it be due to form drag?
Let $\nu \rightarrow 0$ and consider a fluid element lying on the axis
of a bubble or spike.
By symmetry, it will only have a z-component of the velocity.
The z-forces must balance:
$$ \rho \phi g = \pder{P}{z} $$
In the finger, the pressure would be decreasing with $z$.
In the bubble, the pressure would be increasing with $z$.
There would necessarily be a pressure gradient between the head of the finger and the tail of the bubble, and vice versa.
As the aspect ratio exceeded unity, the span-wise pressure gradient would exceed the gravitational forcing.
The resulting span-wise flow would rapidly mix the two fluids, destroying the bubble and spike.

The form of the bubble and spike require the pressure to be reasonably homogeneous span-wise.
Only at the bubble and spike tips do we accept a span-wise flow: the displacement of stationary fluid by the tip.
In other words, the pressure drag is highly localized to the bubble and spike tips.
The flow from tail to tip is attenuated predominately by viscous drag.

\subsection{Historical inconsistency}

The buoyancy-drag model model contains a buoyant term that goes with the bubble volume and a form drag term that goes with the bubble's area.
The model was originally developed to describe multi-mode self-similar flow, in which there is only one length scale, the dominant wavelength $\lambda$.
Consequently, the ratio fo the volume to the surface area is $\lambda^{-1}$, yielding a terminal velocity as a function of $\lambda$.

However, the single-mode RTI has two length scales: in addition to the wavelength $\lambda$ there is the bubble height, $h$.
In other words, single-mode RTI bubbles are cylindrical instead of spherical with an axis length that goes with the bubble height.
The ratio of the volume to surface area is $h^{-1}$, not $\lambda^{-1}$, so there is no terminal velocity.
Only by introducing a drag term that goes with the height $h$, such as skin drag, can a terminal velocity be recovered.
This terminal velocity would be a function of the viscosity, and therefore cannot be described by potential flow.

 % Introduction

%\chapter{Related work}

The shortcomings of models for the single mode low Atwood Rayleigh-Taylor instability, specifically re-acceleration, were first recognized in 2006 by Ramaprabhu et al.~\cite{Ramaprabhu2006} based on numerical simulations.
Experimental confirmation by Wilkinson and Jacobs followed a year later~\cite{Wilkinson2007}.
Since then, multiple attempts have been made to capture re-acceleration in the models.
In this chapter, we review those attempts.

\section{Vortex ring correction of Ramaprabhu}

Ramaprabhu et al. attribute the reacceleration to the formation of a vortex ring at the bubble tip.
They add a term to their buoyancy-drag model representing the centrifugal force per unit volume:
\begin{equation}
\left(\rho_2 g - \rho_1 g\right) + \rho_1 \frac{\omega_0^2 R}{ 4} = \frac{C_d \rho_2 v^2}{\lambda}
\end{equation}
where $\omega_0$ is the average vorticity in the bubble tip.
The model does not provide an evolution equation for $\omega_0$; it is measured from simulations ad-hoc making the model descriptive but not predictive.

\section{Vorticity and viscosity in potential flow}

The effects of the vorticity in the lighter fluid were added to a potential flow model by Banerjee et al~\cite{Banerjee2011}.

 % Introduction

\chapter{Standalone papers}

This thesis contains three standalone papers.
They are logically ordered from the top down.

The first paper contains a new buoyancy-drag model developed to describe the late time behavior in the low-Atwood, moderate Grashof case.
It contains model coefficients that are fit to a data set of direct numerical simulations.
This single author paper will be submitted to for peer review.

The second paper validates those direct numerical simulations against experimental data from Wilkinson and Jacobs~\cite{Wilkinson2007}.
This establishes that the simulations contain the physical processes responsible for re-acceleration and other unexplained late-time phenomena.
The paper takes advantage of the generality of numerical data to explore feature of the flow that were not available to the original experiment, namely the interaction between the bubbles and pressure driving secondary flows in the mid-plane.
This single author paper has been submitted to peer review.

The third paper contains a performance and convergence study of the numerical method and simulation software with the single mode Rayleigh-Taylor problem as a benchmark.
The NekBox code, which was specialized specifically to this project, is shown to be an efficient tool for performing these calculations.
Furthermore, the resolution is selected such that the simulation error is an order smaller than the expected model error, which ensures that the flow is sufficiently but not over resolved.
Maxwell Hutchinson authored all of sections 2 and 4 and the majority of sections 1 and 5.
This paper has been peer reviewed and published in the proceedings of the International Conference on High Performance Computing (ISC).



%\input{Chapters/SingleModeLowRe} % Background Theory 

%\input{Chapters/SingleModeHighRe} % Experimental Setup


%\input{Chapters/SchmidtProjection} % Experiment 1

%\input{Chapters/Wilkinson} % Experiment 2

%\input{Chapters/Implications} % Implications

\chapter{Conclusions}

\section{New model for low-Atwood single mode}

\section{Validation of DNS}

\section{Importance of Schmidt number}

\section{Open questions}

 % Conclusion

%\input{Chapters/Chapter6} % Results and Discussion


%% ----------------------------------------------------------------
% Now begin the Appendices, including them as separate files

\addtocontents{toc}{\vspace{2em}} % Add a gap in the Contents, for aesthetics

\appendix % Cue to tell LaTeX that the following 'chapters' are Appendices

\chapter{Derivations}

\section{Linear stability theory} \alabel{LST}
We start with the incompressible Euler:
$$ \rho \left[ \frac{\partial u}{\partial t} + u \nabla u\right] = -\nabla P + g \rho \qquad \nabla \cdot u = 0$$

We write $P = \bar{P} + \tilde{P}$ and $\rho = \bar\rho + \tilde\rho$:
$$ (\bar\rho +\tilde \rho) \left[ \frac{\partial u}{\partial t} + u \nabla u\right] = -\nabla \tilde{P} + g (\bar\rho+\tilde\rho) \qquad \nabla \cdot u = 0$$

Neglecting terms that go with $|u|^2$ or $u \tilde\rho$ and subtracting $g\bar\rho$:
$$ \bar\rho \frac{\partial u}{\partial t} = -\nabla \tilde{P} + g \tilde\rho \qquad \nabla \cdot u = 0$$

Use a spectral decomposition:
$$ F(x,z,t) \sim \exp(i k x) \exp(i \gamma t) f(z)$$

The momentum equation becomes ($\vec{u} = (u,w)$):
\begin{align}
\bar\rho \gamma u &= - k \tilde{P} \\
i \bar\rho \gamma w &= - \frac{d\tilde{P}}{dz} - g \tilde\rho \\
\end{align}
And incompressibility:
$$ i k u + \frac{dw}{dz} = 0 $$
$$ i \gamma \tilde\rho + w \frac{d\bar\rho}{dz} = 0 $$

Using continuity and the transverse momentum:
$$ \gamma \bar\rho \frac{dw}{dz} = i k^2 \tilde{P}  $$
Take a vertical derviative:
$$ \gamma \frac{d}{dz} \left(\bar\rho \frac{dw}{dz}\right) = i k^2 \frac{d\tilde{P}}{dz}  $$
And use the vertical momentum equation:
$$ \gamma \frac{d}{dz} \left(\bar\rho \frac{dw}{dz}\right) = - i k^2 \left(i \bar\rho \gamma w + g \tilde\rho\right)  $$

Substituting in momentum advection:
$$ \gamma \frac{d}{dz} \left(\bar\rho \frac{dw}{dz}\right) = - i k^2 \left(i \bar\rho \gamma w + \frac{i g w}{\gamma} \frac{d \bar\rho}{dz}\right)  $$
Cleaning up a bit:
$$ \frac{d}{dz} \left(\bar\rho \frac{dw}{dz}\right) = w k^2 \left(\bar\rho   + \frac{ g }{\gamma^2} \frac{d \bar\rho}{dz}\right)  $$

Away from the interface:
$$ \frac{d^2}{dz^2}w  = w k^2  $$
So in the upper fluid:
$$ w \sim \exp(ikx) \exp(i\gamma t) \exp(-kz) $$

Now integrate across the interface $\Delta z$:
$$ \left(\bar\rho \frac{dw}{dz}\right)_2 - \left(\bar\rho \frac{dw}{dz}\right)_1 = w k^2 \bar\rho \Delta z  + \left(w k^2 \frac{ g }{\gamma^2}  \bar\rho\right)_2 - \left(w k^2 \frac{ g }{\gamma^2}  \bar\rho\right)_1  $$
We know $dw/dz \sim \pm k w$: Take $\Delta z \rightarrow 0$:
$$ \bar\rho_2 k w + \bar\rho_1 k w = w k^2 \frac{ g }{\gamma^2}  (\bar\rho_2 - \bar\rho_1) $$
Which gives:
$$ \gamma^2 = k g \frac{\bar\rho_2 - \bar\rho_1}{\bar\rho_2+\bar\rho_1}  \equiv A g k $$

\section{Layzer model} \alabel{Layzer}
Since the flow is invicid and irrotational, Layzer~\cite{Layzer1955} uses the potential flow technique, writing the velocity as the gradient of a scalar potential:
\begin{equation}
v = \nabla \phi
\end{equation}
where 
$v$ is the velocity and 
$\phi$ is the scalar potential.
Incompressibility zeroes the Laplacian of the potential:
\begin{equation}
\nabla^2 \phi = 0
\end{equation}
A Bernoulli equation is used model the interface:
\begin{equation} \elabel{layzer_jump}
\der{\phi}{t}(r, \eta(r,t), t) + \frac{1}{2} \left(\left(\pder{\phi}{z}\right)^2(r, \eta(r,t), t) +\left(\pder{\phi}{r}\right)^2(r, \eta(r,t), t)\right) + g \eta(r,t) = f(t)
\end{equation}
where 
$\eta(r,t)$ is the height of the interface,
$g$ is the gravitational acceleration, and 
$f(t)$ is an arbitrary function of time but not space.
The flow is axially symmetric with a vanishing radial component at transverse walls and vanishing vertical component far away from the bubble:
\begin{equation}
\pder{\phi}{r}(R,z,t) = 0 \qquad \pder{\phi}{z}(r, \pm \infty, t) = 0
\end{equation}
Finally, the fluid advects the interface:
\begin{equation} \elabel{layzer_adv}
\der{\eta}{t}(r,t) = \pder{\phi}{z}(r, \eta(r,t), t) - \pder{\phi}{r}(r, \eta(r,t), t) \der{\eta}{r}(r,t)
\end{equation}

The simplest non-trivial potential that fits the boundary conditions is:
\begin{equation}
\phi(r,z,t) = F(t) J_0(r) e^{-z}
\end{equation}

Substituting this potential into \eref{layzer_jump} and \eref{layzer_adv}:
\begin{align}
f(t) &= F'(t) J_0(r) e^{-\eta(r,t)} + \frac{F^2(t) e^{-2\eta(r,t)}}{2} \left[ J_0^2(r)  + J_1^2(r) \right] + g \eta(r,t) \\
\der{\eta}{t}(r,t) &=  -F(t) J_0(r)e^{-\eta(r,t)}  + F(t) J_1(r) e^{-\eta(r,t)} \der{\eta}{r}(r,t)
\end{align}

Expanding the bubble tip to the 2nd order as $\eta(r,t) = \eta_0(t) + r^2 \eta_2(t)$:
\begin{align}
f(t) &= F'(t) J_0(r) e^{-\eta_0(t)-\eta_2(t)r^2} + \frac{F^2(t) e^{-2\eta_0(t)-2\eta_2(t)r^2}}{2} \left[ J_0^2(r)  + J_1^2(r) \right] + g \eta_0(t) + g \eta_2(t)r^2 \\
\dot\eta_0 + r^2 \dot\eta_2 &=  -F(t) J_0(r)e^{-\eta_0(t) - \eta_2 r^2}  + 2 F(t) J_1(r) e^{-\eta_0(t) - \eta_2(t)r^2} r \eta_2(t) 
\end{align}

Keeping terms up to $r^2$:
\begin{align}
f(t) &= F'(t) \left(1-r^2/4\right) (1-\eta_2(t)r^2) e^{-\eta_0(t)} + \frac{F^2(t) (1-2\eta_2(t)r^2) e^{-2\eta_0(t)}}{2} \left[1-r^2/4\right] + g \eta_0(t) + g \eta_2(t)r^2 \\
\dot\eta_0 + r^2 \dot\eta_2 &= (-1+r^2/4 + r^2 \eta_2(t)) F(t) (1-\eta_2(t)r^2) e^{-\eta_0(t)} 
\end{align}

We collect the terms based on their $r$-dependence:
\begin{align}
f(t) &= F'(t) e^{-\eta_0(t)} + \frac{F^2(t) e^{-2\eta_0(t)}}{2} + g \eta_0(t)  \\
0 &= - F'(t)(\eta_2(t)+ 1/4) e^{-\eta_0(t)} - \frac{F^2(t) e^{-2\eta_0(t)}}{2}(2\eta_2(t)+1/4) + g \eta_2(t) \\
\dot\eta_0  &= - F(t)  e^{-\eta_0(t)}  \\
\dot\eta_2 &= (1/4 + 2\eta_2(t)) F(t)  e^{-\eta_0(t)} 
\end{align}
Cleaning up:
\begin{align}
f(t) &= F'(t) e^{-\eta_0(t)} + \frac{F^2(t) e^{-2\eta_0(t)}}{2} + g \eta_0(t)  \\
0 &= - F'(t) e^{-\eta_0(t)} (4\eta_2(t)+ 1)- \frac{F^2(t) e^{-2\eta_0(t)}}{2}(8\eta_2(t)+1) + 4g \eta_2(t) \\
\dot\eta_0  &= - F(t)  e^{-\eta_0(t)}  \\
\dot\eta_2 &= (1/4 + 2\eta_2(t)) F(t)  e^{-\eta_0(t)} 
\end{align}

At late time, $\eta_2 = -1/8$ such that $\dot\eta_2 = 0$:
\begin{align}
f(t) &= F'(t) e^{-\eta_0(t)} + \frac{F^2(t) e^{-2\eta_0(t)}}{2} + g \eta_0(t)  \\
0 &=  F'(t) e^{-\eta_0(t)}  + g  \\
\dot\eta_0  &= - F(t)  e^{-\eta_0(t)}  \\
\end{align}

Using the evolution equation for $\eta_0$ to eliminate $F$:
\begin{align}
F(t) &= -\dot\eta_0 e^{\eta_0(t)}    \\
0 &= -\left(\ddot\eta_0(t)+(\dot\eta_0(t))^2\right) + g 
\end{align}

The solution to this second order nonlinear ordinary differential equation is:
\begin{equation}
\eta_0(t) = \log\left( \exp\left[2 \sqrt{g} (c_1 + t) \right] + 1 \right) - \sqrt{g}t + c_2
\end{equation}

For large $t$:
\begin{equation}
\eta_0(t>>1) \approx  \sqrt{g}t 
\end{equation}

Substituting back $r = \frac{R}{\beta_1}$, where $J_1(\beta_1) = 0$:
\begin{equation}
V = \sqrt{\frac{g R}{\beta_1}} \approx 0.511 \sqrt{g R}
\end{equation}

	% Appendix Title

\chapter{Observables}

\section{Incompressible flows}

\subsection{Energy balance}

The kinetic energy of a fluid flow is defined as:
\begin{equation} \elabel{kinetic-energy}
T = \int_\Omega \left[\frac{1}{2} \rho u^2 \right] dV
\end{equation}

The gravitational potential is:
\begin{equation} \elabel{grav-energy}
V_g = \int_\Omega \left[ \rho  g \cdot r \right] dV
\end{equation}

The rate of local dissipation:
\begin{equation} \elabel{diss-vol-rate}
\Phi = \nu \left[ 2 \left(\pder{u_i}{x_i}\right)^2 + \left(u_i u_j \epsilon_{i,j,k}\right) \left(u_I u_J \epsilon_{I,J,k}\right) \right]
\end{equation}
 % Appendix Title

%\chapter{Adjoints}

\section{Example: Burger's equation}

Let the objective function be the total kinetic energy, $T = \frac{1}{2} \int u^2 dx$.
This can be written as:
\begin{align} \elabel{burg_T}
T &= \frac{1}{2} \int u(x,t)^2 dx \\
&= \int dx \\
&\left[-2 \nu \pder{}{x} \ln \left\{(4 \pi \nu t)^{-1/2} \int_{-\infty}^\infty \exp\left[-\frac{(x-x')^2}{4\nu t} - \frac{1}{2\nu} \int_0^{x'} u(x'',0) dx''\right] dx'\right\} \right]^2 \\
\end{align}

We use the chain rule:
\begin{equation}
\pder{T(t)}{u(x,0)} = \pder{T}{u(y,t)} \pder{u(y,t)}{\phi(z,t)} \pder{\phi(z,t)}{\phi(w,0)} \pder{\phi(w,0)}{u(x,0} 
\end{equation}
We can write each of these down:
\begin{align*}
\pder{T}{u(y,t)} &= u(y,t) \\
\pder{u(y,t)}{\phi(z,t)} &= -\frac{2 \nu}{\phi(z,t)} \left[\ppder{\phi(z,t)}{z} - \frac{1}{\phi} \pder{\phi(z,t)}{z}\right] \delta(y-z) \\
\pder{\phi(z,t)}{\phi(w,0)} &= \frac{1}{\sqrt{4 \nu \pi t}} e^{-(z-w)^2/(4 \nu t)} \\
\pder{\phi(w,0)}{u(x,0)} &= - \frac{\phi(x,0)}{2 \nu} \Theta(w-x) 
\end{align*}
 % Appendix Title

\addtocontents{toc}{\vspace{2em}}  % Add a gap in the Contents, for aesthetics
\backmatter

%% ----------------------------------------------------------------
\label{Bibliography}
\lhead{\emph{Bibliography}}  % Change the left side page header to "Bibliography"
\bibliographystyle{unsrtnat}  % Use the "unsrtnat" BibTeX style for formatting the Bibliography
\bibliography{library}  % The references (bibliography) information are stored in the file named "Bibliography.bib"

\end{document}  % The End
%% ----------------------------------------------------------------
