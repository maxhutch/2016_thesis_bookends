%% ----------------------------------------------------------------
%% Thesis.tex -- MAIN FILE (the one that you compile with LaTeX)
%% ---------------------------------------------------------------- 

% Set up the document
\documentclass[a4paper, 11pt, oneside]{Thesis}  % Use the "Thesis" style, based on the ECS Thesis style by Steve Gunn
\graphicspath{Figures/}  % Location of the graphics files (set up for graphics to be in PDF format)

% Include any extra LaTeX packages required
\usepackage[square, numbers, comma, sort&compress]{natbib}  % Use the "Natbib" style for the references in the Bibliography
\usepackage{verbatim}  % Needed for the "comment" environment to make LaTeX comments
\usepackage{vector}  % Allows "\bvec{}" and "\buvec{}" for "blackboard" style bold vectors in maths
\usepackage{mysty}
\hypersetup{urlcolor=blue, colorlinks=true}  % Colours hyperlinks in blue, but this can be distracting if there are many links.

%% ----------------------------------------------------------------
\begin{document}
\frontmatter      % Begin Roman style (i, ii, iii, iv...) page numbering

% Set up the Title Page
\title  {Data driven modeling of the low-Atwood single-mode Rayleigh-Taylor instability}
\authors  {\texorpdfstring
            {\href{maxhutch@uchicago.edu}{Maxwell Hutchinson}}
            {Maxwell Hutchinson}
            }
\addresses  {\groupname\\\deptname\\\univname}  % Do not change this here, instead these must be set in the "Thesis.cls" file, please look through it instead
\date       {\today}
\subject    {}
\keywords   {}

\maketitle
%% ----------------------------------------------------------------

\setstretch{1.3}  % It is better to have smaller font and larger line spacing than the other way round

% Define the page headers using the FancyHdr package and set up for one-sided printing
\fancyhead{}  % Clears all page headers and footers
\rhead{\thepage}  % Sets the right side header to show the page number
\lhead{}  % Clears the left side page header

\pagestyle{fancy}  % Finally, use the "fancy" page style to implement the FancyHdr headers

%% ----------------------------------------------------------------
% Declaration Page required for the Thesis, your institution may give you a different text to place here
\Declaration{

\addtocontents{toc}{\vspace{1em}}  % Add a gap in the Contents, for aesthetics

I, Maxwell Hutchinson, declare that this thesis titled, `Data driven modeling of the low-Atwood single-mode Rayleigh-Taylor instability' and the work presented in it are my own. I confirm that:

\begin{itemize} 
\item[\tiny{$\blacksquare$}] This work was done wholly or mainly while in candidature for a research degree at this University.
 
\item[\tiny{$\blacksquare$}] Where any part of this thesis has previously been submitted for a degree or any other qualification at this University or any other institution, this has been clearly stated.
 
\item[\tiny{$\blacksquare$}] Where I have consulted the published work of others, this is always clearly attributed.
 
\item[\tiny{$\blacksquare$}] Where I have quoted from the work of others, the source is always given. With the exception of such quotations, this thesis is entirely my own work.
 
\item[\tiny{$\blacksquare$}] I have acknowledged all main sources of help.
 
\item[\tiny{$\blacksquare$}] Where the thesis is based on work done by myself jointly with others, I have made clear exactly what was done by others and what I have contributed myself.
\\
\end{itemize}
 
 
Signed:\\
\rule[1em]{25em}{0.5pt}  % This prints a line for the signature
 
Date:\\
\rule[1em]{25em}{0.5pt}  % This prints a line to write the date
}
\clearpage  % Declaration ended, now start a new page

%% ----------------------------------------------------------------
% The "Funny Quote Page"
\begin{comment}
\pagestyle{empty}  % No headers or footers for the following pages

\null\vfill
% Now comes the "Funny Quote", written in italics
\textit{``Bubbler morghulis -- all bubbles must die''}

\begin{flushright}
No one
\end{flushright}

\vfill\vfill\vfill\vfill\vfill\vfill\null
\end{comment}
\clearpage  % Funny Quote page ended, start a new page
%% ----------------------------------------------------------------

% The Abstract Page
\addtotoc{Abstract}  % Add the "Abstract" page entry to the Contents
\abstract{
\addtocontents{toc}{\vspace{1em}}  % Add a gap in the Contents, for aesthetics
The Rayleigh-Taylor instability is one of the most common and well studied phenomena in fluid dynamics. 
%, appearing and impacting such diverse flows as the doubly diffusive salt finger, nuclear flames in type 1a supernovae, and ablator mixing in inertial confinement fusion.
Despite research dating to 
%its characterization by Lord Rayleigh in 
the late 19th century, the non-linear dynamics 
%of bubbles originating 
of the interfacial instability are still not fully understood, particularly in the case when the two fluids have nearly the same density.
It was recently demonstrated in this, the low-Atwood regime, that the idealized single-mode problem departs from established potential flow models in the form of a re-acceleration beyond the predicted terminal interface velocity.
This thesis is an attempt to model that re-acceleration and, more broadly, the late time dynamics of the single-mode low-Atwood Rayleigh-Taylor instability.

The approach taken here is based on buoyancy-drag models, which express a force balance between buoyancy and parasitic drag.
The dynamical buoyancy-drag model is supplemented with a mixing model that dilutes the buoyant force over time.
These models are written deliberately generally, with 8 unique coefficients.
Three of these coefficients are solved for by equating the early time behavior with that of well established linear theories.
The remaining 5 coefficients are estimated by relating them to drag coefficients, friction factors, and geometric ratios in the interface shape.

To evaluate the model and compute the 5 unknown coefficients more precisely, a set of direct numerical simulations are performed over the relevant parameter space.
These simulations are first validated against experimental data.
Then they are shown to converge and their resolutions are chosen such as to minimize computational cost given the accuracy scale of the model.
The 5 coefficients are fit to the resulting data set, and the model achieves better than 2\% error in the bubble height and 4\% error in the volume of mixed fluid.
Three coefficients are nominally independent of the parameterization of the problem, while two are shown to vary with the Rayleigh number and the diffusivity.

%The approach taken in this thesis is a starting point for data-driven modeling of model fluid dynamics problems.
%The model could be augmented with a vorticity term, possibly capturing re-acceleration.
%Many of the simulated cases are incomplete in that the Rayleigh-Taylor front reached the top and bottom boundaries of the simulated domain.
%The computational cost is shown to go with the Rayleigh number to the 6th power, making the compleition of these cases expensive.
}

\clearpage  % Abstract ended, start a new page
%% ----------------------------------------------------------------

\setstretch{1.3}  % Reset the line-spacing to 1.3 for body text (if it has changed)

% The Acknowledgements page, for thanking everyone
\acknowledgements{
\addtocontents{toc}{\vspace{1em}}  % Add a gap in the Contents, for aesthetics

I would like to thank Robert Rosner for letting me chart my course through this thesis and my doctoral research more broadly.
I thank Aleksandr Obabko and Paul Fischer for supporting Nek5000 and working with me through the numerics.
I also thank Oana Marin and Michel Schanen for their excitement and Elia Merzari and Ron Rahaman for their support.
I thank Alexander Heinecke and Scott Parker for their help tuning the code to run on supercomputers.
I thank Elizabeth Hicks for helpful comments and advice.

For computer time, this research partially used the resources of the
Supercomputing Laboratory at King Abdullah University of Science \& Technology
 (KAUST) in Thuwal, Saudi Arabia.
This research used resources of the Argonne Leadership Computing Facility, which
is a DOE Office of Science User Facility supported under Contract DE-AC02-06CH11357.

}
\clearpage  % End of the Acknowledgements
%% ----------------------------------------------------------------

\pagestyle{fancy}  %The page style headers have been "empty" all this time, now use the "fancy" headers as defined before to bring them back


%% ----------------------------------------------------------------
\lhead{\emph{Contents}}  % Set the left side page header to "Contents"
\tableofcontents  % Write out the Table of Contents

\begin{comment}
%% ----------------------------------------------------------------
\lhead{\emph{List of Figures}}  % Set the left side page header to "List if Figures"
\listoffigures  % Write out the List of Figures

%% ----------------------------------------------------------------
\lhead{\emph{List of Tables}}  % Set the left side page header to "List of Tables"
\listoftables  % Write out the List of Tables

%% ----------------------------------------------------------------
\setstretch{1.5}  % Set the line spacing to 1.5, this makes the following tables easier to read
\clearpage  % Start a new page
\lhead{\emph{Abbreviations}}  % Set the left side page header to "Abbreviations"
\listofsymbols{ll}  % Include a list of Abbreviations (a table of two columns)
{
% \textbf{Acronym} & \textbf{W}hat (it) \textbf{S}tands \textbf{F}or \\
\textbf{LAH} & \textbf{L}ist \textbf{A}bbreviations \textbf{H}ere \\

}
\end{comment}

%% ----------------------------------------------------------------
%\clearpage  % Start a new page
%\lhead{\emph{Physical Constants}}  % Set the left side page header to "Physical Constants"
%\listofconstants{lrcl}  % Include a list of Physical Constants (a four column table)
%{
% Constant Name & Symbol & = & Constant Value (with units) \\
%Speed of Light & $c$ & $=$ & $2.997\ 924\ 58\times10^{8}\ \mbox{ms}^{-\mbox{s}}$ (exact)\\
%
%}

%% ----------------------------------------------------------------
\clearpage  %Start a new page
\lhead{\emph{Symbols}}  % Set the left side page header to "Symbols"
\listofnomenclature{lll}  % Include a list of Symbols (a three column table)
{
% symbol & name & unit \\
D & Mass diffusivity & m$^2$ / s \\
g & Frame acceleration &  m / s$^2$ \\
P & Pressure & kg / (m s$^2$)\\
$\lambda$ & Single-mode wavelength & m \\
$\nu$ & Kinematic viscosity & m$^2$ / s \\
$\rho_h$ & Higher density & kg / m$^3$ \\
$\rho_l$ & Lower density & kg / m$^3$  \\
\hline 
A & Atwood number & $\frac{\rho_h - \rho_l}{\rho_h + \rho_l}$ \\
Gr & Grashof number & $ \frac{A g \lambda^3}{\nu^2}$ \\
Re$_p$ & Perturbation Reynolds number & $\sqrt{\text{Gr}}$ \\
Ra & Rayleigh number & $ \frac{A g \lambda^3}{\nu D}$ \\
Sc & Schmidt number & $\nu$ / D  \\
}
%% ----------------------------------------------------------------
% End of the pre-able, contents and lists of things
% Begin the Dedication page

\setstretch{1.3}  % Return the line spacing back to 1.3

\pagestyle{empty}  % Page style needs to be empty for this page
\dedicatory{Dedicated to Tracey Ziev}

\addtocontents{toc}{\vspace{2em}}  % Add a gap in the Contents, for aesthetics


%% ----------------------------------------------------------------
\mainmatter	  % Begin normal, numeric (1,2,3...) page numbering
\pagestyle{fancy}  % Return the page headers back to the "fancy" style

% Include the chapters of the thesis, as separate files
% Just uncomment the lines as you write the chapters

\lhead{}  % Set the left side page header to "Symbols"
\chapter{Introduction}

\section{Formal definition}
The Rayleigh-Taylor instability occurs when the pressure and density gradients are in opposition:
\begin{equation}
(\nabla P)(\nabla \rho) < 0
\end{equation}
The canonical example of the Rayleigh-Taylor instability is a heavy fluid superposed over a lighter one in a gravitational field.
The standard terminology is based on this case.

Consider a horizontal planar interface at $z=0$ between two fluids of densities $\rho_h > \rho_l$, corresponding to the density of the heavier and lighter fluid, respectively.
The denser fluid is at $z > 0$ and the lighter at $z < 0$, and a gravitational acceleration $-g \hat{z}$.
In equilibrium, the pressure must balance the gravitational force:
\begin{equation}
P(x,y,z) = \begin{cases}- \rho_h g z + C& \text{ if } z > 0 \\
                        - \rho_l g z + C& \text{ else}
           \end{cases}
\end{equation}
A small perturbation is introduced, moving some heavy fluid below the interface and some light fluid above it but preserving the horizontal stratification of the pressure.
The forcing on the heavier fluid below the interface is then:
\begin{equation}\elabel{force1}
\sum F = - \nabla P + F_g = \rho_l g  - \rho_h g < 0,
\end{equation}
so the heavier fluid is forced downwards through the lighter fluid.
Conversely, the forcing on the lighter fluid above the interface is:
\begin{equation} \elabel{force2}
\sum F = - \nabla P + F_g = \rho_l g  - \rho_h g > 0,
\end{equation}
so the lighter fluid is forced upwards through the heavier fluid.
Perturbations in the interface grow, so the configuration is unstable.

\section{Instances and motivation}
The Rayleigh-Taylor instability is present in both natural and constructed systems at many scales.
This section describes a few such systems of particular importance.

\paragraph{Type Ia Supernovae}
In type Ia supernovae (SNe Ia), a white dwarf spontaneously ignites as its mass crosses the Chandrasekhar mass due to accretion from another source.
The fusion flame originates near the center of the star and burns outward.
The hot ash trailing the flame is lighter than the fuel, creating a Rayleigh-Taylor unstable flame front.
The primary RT instability and secondary Kelvin-Helmholtz instabilities wrinkle the flame front, enhancing mixing, burn rate, and flame speed~\cite{Zingale2005}.
These properties affect the rate of energy release and ejecta velocity, which can be observed.
The use of SNe Ia as standard candles underlines interest in their description.

\paragraph{Salt fingers and the thermohaline staircase}
Salt fingers are an instance of the Rayleigh-Taylor instability set up by a difference in the diffusivity of two mass carriers~\cite{Stern1969, Linden1973}.
Consider a fluid in which the density perturbation is a linear combination of two fields.
Let one of the fields have a stabilizing gradient and the other a destabilizing one.
If the destabilizing field is less diffusive than the stabilizing one, the system is unstable.
A parcel of fluid perturbed from its equilibrium height will equilibrate with the stabilizing field before the destabilizing one, resulting in a buoyant force that pushes the parcel further from vertical equilibrium.

In the oceanic case, the stabilizing field is temperature and the destabilizing field is salinity.
Near the surface, evaporation perturbs the salinity field creating parcels of salty dense fluid.
As they sink, the parcels cool more quickly than they diffuse salt, further increasing their density.
This flow drives tall vertical convective cells, called salt fingers, that mixes the oceans.

The same doubly diffusive buoyant process drives short broad convective cells at greater depths.
Observed as thermohaline staircase~\cite{Tait1971}, a series of sharp steps in the salinity and temperature vs depth, these cells extend occur at depths greater than a kilometer and extend for thousands of square miles.

\paragraph{ICF}
Inertial confinement fusion (ICF) is a fusion technique that confines hot dense plasma temporarily via an implosion, rather than magnetic field lines.
In ICF, the cryogenic hydrogen is coated with a plastic ablator forming very small hollow spheres, or microcapsules.
The microcapsules are through a cylindrical hohlraum.
When the capsule is at the center of the hohlraum, the hohlraum is illuminated with a high energy burst of laser light and radiates x-rays.
The x-ray radiation is absorbed by the plastic ablator causing it to rapidly expand and blow off the microcapsule, creating an implosion in the hydrogen fuel.

The implosion isn't perfectly uniform; the x-ray radiation provides non-uniform acceleration and there are asymmetries in the microcapsule.
These perturbations are Rayleigh-Taylor unstable: the dense plastic ablator is being accelerated through lighter hydrogen fuel.
The carbon in the ablator is much better at radiating energy than the hydrogen, so Rayleigh-Taylor mixing cool the fuel, preventing ignition.

\section{Terminology}

In this section, we introduce common Rayleigh-Taylor terminology that will be used throughout the thesis.

\paragraph{Atwood number}
The Atwood number characterizes the density contrast:
\begin{equation} \elabel{atwood}
A = \frac{\rho_1 - \rho_2}{\rho_1 + \rho_2} \in (-1,1)
\end{equation}
where $\rho_1 > \rho_2$ corresponds to positive Atwood number.
Negative Atwood numbers result in stable oscillating interfaces.
There are three distinct regimes of unstable Atwood numbers: high, low, and moderate.

At high Atwood number, e.g. air and water, the flow is approximated in the limit of unit Atwood number.
The internal dynamics of the light fluid are decoupled from the heavy fluid, and the transfer of momentum through the interface into the dense fluid can be neglected.
In this regime, the flow of the dense fluid is nearly irrotational and potential flow models are reasonably accurate.
% When is it high enough?

At low Atwood number, e.g. salt water and fresh water, the flow is approximated in the limit of zero Atwood number.
The governing equations can be linearised, leading to the Boussinesq approximation: only terms with the product of the Atwood number and acceleration remain.
The behavior of the bubbles and spikes are symmetric given symmetric initial conditions.
It is often possible to treat the true two-phase problem as a single phase with an active scalar.
% When is it Boussinsq?

At moderate Atwood number, e.g. oil and water, not only is enough vorticity is generated to undermine potential flow models but also the density difference breaks the Boussinesq approximation and bubble-spike symmetry.
These are truly two-phase problems in that the fluids are strongly coupled and have different governing parameters, e.g. viscosity.

\paragraph{Boussinesq approximation}
When the density contrast, $\Delta \rho$ is small compared to the average density $\bar{\rho}$, then 
the governing equations can be linearized with respect to density.
This is known as the Boussinesq approximation and has the primary effect of neglecting differences in the inertia of the two fluids.

\paragraph{Constant property}
In the spirit of the Boussinesq approximation, which neglects differences in the inertial of the two fluids, one can further assume the two fluids share all other material properties, such as viscosity.
When the flow is both Boussinesq and has constant properties, the density can be modeled as an active scalar with a buoyant forcing term.

\paragraph{Miscible interface}
In most cases where the Boussinesq and constant property approximations are valid, the interface between the two fluids is miscible.
For example, low-concentration solutions in a common solvent have interfaces governed by a diffusion coefficient, $D$, and small temperature gradients are governed by a thermal diffusivity $\alpha$.
In the limit where $D$ or $\alpha$ goes to zero, the interface is immiscible but there is no surface tension.
If there is no surface tension, the active scalar is governed by an advection-diffusion equation.

\paragraph{Incompressible}
The problem is said to be incompressible when each of the two fluids are.
Formally, this means the flow is divergence free, $\nabla \cdot u = 0$ and occurs when the Mach number is small.
In simulations, incompressibility has the effect of integrating out acoustic waves, significantly reducing the need to resolve small time-scales in the flow.

\paragraph{Single-mode}
The single mode Rayleigh-Taylor instability constrains the initial perturbation of the interface to a single pure frequency.
In 3D, this typically implies two orthogonal wavevectors with the same wavelength.
For the miscible constant property Boussinesq Rayleigh-Taylor instability, the single mode initial condition is:
\begin{equation}
\phi(x,y,z,t=0) = A ~ \text{erf}\left(\frac{z + a_0 \cos(2 \pi x / \lambda) \cos(2 \pi y/\lambda)}{\delta}\right),
\end{equation}
where $a_0$ is the perturbation height,
$\lambda$ is the wavelength, and
$\delta$ is the interface thickness.
Typically, the perturbation height and interface thickness are chosen to initially be much smaller than the wavelength, $a_0, \delta << \lambda$.

\paragraph{Multi-mode}
The multi-mode Rayleigh-Taylor instability generically refers to the presence of more than one, and often many, wavelengths in the initial condition.
Experimentally, the multi-mode initial condition is usually `natural`, in the sense that it is not deliberately perturbed and instead is due to natural noise sources.
Computationally, the multi-mode condition is written as a sum of single modes with spectra $z_0(k) \sim |k|^{-p}$ for $p = 2$, $3/2$, or other rational factors.
These modes are randomized within the spectral envelop.

\paragraph{Governing equations}
Given a miscible interface between two Boussinesq, constant property incompressible fluids, the governing equations can be recast in terms of an active scalar:
\begin{align}
\der{u}{t} + u \cdot \nabla u &= \nu \nabla^2 u - \nabla P - A g \phi \hat{z}\\
\der{\phi}{t} + u \cdot \nabla \phi &= D \nabla^2 \phi \\
\nabla \cdot u  &= 0,
\end{align}
where $u$ is the fluid velocity, 
$\phi$ is the active scalar that represents small mass differences,
and the acceleration is aligned vertically with $\hat{z}$.

These equations have three governing parameters; $(Ag)$, $\nu$, and $D$; which admit the construction of two dimensionless numbers:
\begin{equation}
\text{Grashof} = \frac{Ag L^3}{\nu^2}
\end{equation}
\begin{equation}
\text{Schmidt} = \frac{\nu}{D}
\end{equation}
If the initial and boundary conditions can be characterized by a single length scale, then these two parameters uniquely identify the system.
In the single-mode case, this length is the wavelength $\lambda$.

It is convenient to introduce another dimensionless number, the Rayleigh number:
\begin{equation}
\text{Rayleigh} = \text{Grashof} \cdot \text{Schmidt} = \frac{A g L^3}{\nu D},
\end{equation}
with which many mixing related quantities scale.
The three dimensionless numbers are abbreviated Gr, Sc, and Ra, respectively.

It is difficult to define a Reynolds number in the absense of a velocity parameter.
The square root of the Grashof number, which has the same scaling with the viscosity, is used instead.
It is sometimes called the perturbation Reynolds number, $\text{Re}_p = \sqrt{\text{Gr}}$.

\paragraph{Bubble height and spike depth}

Experiments and theories focus on the observation and explanation of a set of observables that are much smaller than the full degrees of freedom of the system.
The selection of these observables is informed by practical application and methodological analogy.

The bubble height refers to the distance beyond the initial interface that the bubble front has traveled, as a function of time.
The standard experimental definition projects the density onto a line normal to the initial interface and defines the front interface as the point at which the density is at its 99th or 95th percentile.
More formally:
\begin{equation}
H_p[\epsilon] = \sup \left\{z : \int \tilde\rho(x,y,z) dx dy < (1-\epsilon) \int \tilde\rho(x,y,\infty) dx dy \right\},
\end{equation}
where $\tilde\rho$ is the deviation from the mean density, $\tilde\rho = \rho - \bar\rho$.

If the fluids are miscible, this definition depends on the rate of diffusion across the interface.
To avoid dependence on diffusion across the interface, we can base the bubble height on a measurement of the equi-molar interface, which is stationary under diffusion.
To pick out the interface, we take a span-wise maximum instead of a span-wise sum:
\begin{equation}
H_m[\epsilon] = \sup \left\{z : \max_{x,y} \tilde\rho(x,y,z) < 0 \right\}
\end{equation}

However, diffusive mixing across the sides of an elongated bubble dilute as it grows.
This dilution, which is observed as a linear profile in the span-wise maximum instead of an error function profile, can dip below $\tilde\rho = 0$, at which point the growth of $H_m$ is also influenced by mixing.

In the absence of this affect, $H_{m}$ tracks not only the equi-molar surface but also the inflection point in the profile $\max_{x,y} \tilde \rho$.
This inflection point closely tracks the equi-molar surface at low diffusivity but is robust to diffusion across the bubble, remaining in the center of the error function region of the maximum density profile.
Formally, this definition is:
\begin{equation}
H_i[\epsilon] = \sup \left\{z : \frac{d^2}{dz^2} \max_{x,y} \tilde\rho(x,y,z) = 0 \right\}
\end{equation}

Equivalent definitions can be given for the spike depth.
If the flow is Boussinesq and the initial condition is symmetric, the bubble height and spike depth can be averaged.

\paragraph{Mixing width}

In some application, the interpenetration of the two fluids is secondary to their mixing.
To measure the mixed-ness of the flow, we map the pure fluids to unity, mixed fluids to zero, and integrate.
First, we define a normalized scalar measure of the density as:
\begin{equation}
\phi = \frac{2\rho - 2\bar\rho}{\rho_1 - \rho_2} \in \left[-1,1\right],
\end{equation}
then the purity of a fluid volume is $|\phi|$.
The mix volume is simply the integral:
\begin{equation}
\Theta(t) = \int \left(1 - \left|\phi(x,y,z,t)\right|\right) dV,
\end{equation}
and the mixing width $\theta = \Theta / A$, where $A$ is the span-wise extent of the volume integral.


\begin{comment}
\section{Stages}

\subsection{Linear growth} 
When the amplitude of the surface perturbation is small compared to its wave-length, the perturbation grows exponentially.
The growth rate is seen to depend on the forcing, wave-length, viscosity, diffusivity, and interface thickness.
This stage can be treated with linear and weakly-nonlinear theories, as discussed further in \sref{linear}.


\subsection{Single-mode stagnation}
As the growing Rayleigh-Taylor modes saturate, the acceleration of the bubble tip decreases.
The stagnation stage of the single-mode Rayleigh-Taylor instability is defined as a stage of the flow that exhibits nearly constant bubble velocity.
It had been believed that the first such stage was terminal, i.e. steady state.
It has recently been shown, via both simulation~\cite{Ramaprabhu2006} and experiment~\cite{Wilkinson2007}, that at low Atwood number and high Reynolds number the flow continues to develop into re-acceleration stage.
For this reason, the nearly constant velocity is termed the \textit{stagnation velocity}.
The primary observable of a stagnation stage is the stagnation velocity, but there is secondary interest in the velocity profile through the transition from the weakly non-linear to stagnation stages and in the bubble height at which stagnation velocity is reached.

\subsection{Single-mode late time}
For low Atwood and high Reynolds numbers, the stagnation stage is followed by a reacceleration stage around unity aspect ratio.
Reacceleration has been seen in both numerical simulations \cite{Ramaprabhu2006, Ramaprabhu2012, Wei2012} and experiments{Wilkinson2007}.
Following the immediate reacceleration, there is disagreement as to whether or not the single mode problem ultimatly reaches a terminal stage.
Ramaprabhu \etal suggest that the re-acceleration is transient, with the bubble velocity ultimately returning to a potential-flow-like value.
Wei and Livescu, on the other hand, suggest the terminal stage to be one of `chaotic development` with quadratic growth dynamics.
Unfortunately, the late-time stages have yet to be accessed experimentally.

\subsection{Multi-mode}
Under natural multi-mode initial conditions, the linear growth modes couple as the aspect ratio increases, forming bubble and spike structures.
The bubbles and spikes interact with one another, competetively and constructively, leading to successively larger structures.
It is generally observed that the multi-mode aspect ratio, $D_b / h$, remains nearly constant.
\end{comment}
 % Introduction

\chapter{Background}

The study of the Rayleigh-Taylor instability (RTI) is primarily interested in evolution of the interface, that is the rate of penetration of the light fluid into the dense one and vice versa.
While the volumetric mixing rate is relevant in some contexts, most flows have relatively low diffusivity, i.e., high Prandtl and Schmidt number, so mixing is dominated by transport rather than diffusion.
In this section, we review approaches to modeling the evolution of the interface and experimental efforts to validate those models.

\section{Linear and weakly non-linear models} %\slabel{linear}

The earliest models of the Rayleigh-Taylor instability were based on a linearization of the governing equations around small perturbations in the interface.
Recently, with the aid of computational algebra, it has become possible to retain higher order terms in the expansion, demonstrating mode coupling and saturation amplitudes.
However, even high order expansions fail as the interface loses analyticity.

\subsection{Lord Rayleigh's linear model}

Lord Rayleigh considered a sinusoidal perturbation of an incompressible, inviscid, immiscible, quiescent stratified interface~\cite{Rayleigh1883}.
When the amplitude is small compared to the wavelength, the continuity and momentum equations can be linearised:
\begin{align}
\left(\bar\rho + \tilde\rho\right) \left[\der{u}{t} + u \nabla u \right] &= - \nabla{\tilde{P}} + g(\bar\rho + \tilde\rho),\\
\nabla \cdot u &= 0,
\end{align}
where $u, \tilde\rho, \tilde{P}$ are the perturbation velocity, density, and pressure.
They are small and their products neglected.
The solution is sinusoidal with an exponential exponential with a growth rate:
\begin{equation} \elabel{simple_growth}
	w \sim e^{i k x} e^{-k z} e^{\gamma t}, \qquad  \gamma^2 = A g k,
\end{equation}
where $w$ is the vertical component of the velocity, 
$g$ is the acceleration experienced by the fluid, and
$k = 2 \pi / \lambda$ is the wave-number of the perturbation.
Positive Atwood numbers correspond to unstable density stratifications, which grow exponentially.
Negative Atwood numbers correspond to stable density stratifications, which oscillate.

\subsection{Viscous and diffusive linear models}
Chandrasekhar~\cite{Chandrasekhar1955} and Hide~\cite{Hide1955} generalized the linear theory to viscous fluids by including an isotropic incompressible Newtonian shear stress.
Chandrasekhar worked out the uniform constant property case, $\mu_1 = \mu_2$, and Hide included an approximate combination of distinct viscosities.
Here, we are concerned with the simpler uniform constant property case:
\begin{equation} \elabel{visc_growth}
  \gamma = \sqrt{A g k + \nu^2 k^4} - \nu k^2,
\end{equation} 
where 
$\nu$ is the kinematic viscosity.
Note that the viscous growth rate has a fastest growing mode at finite wavenumber and that all wavenumbers have positive growth rates.

LeLevier \etal~\cite{LeLevier1955} generalized the linear theory to continuous density gradients, specifically exponentially smoothed profiles for the form $\bar\rho \pm e^{\mp K z} \delta\rho$:
\begin{equation}
\gamma = \sqrt{\frac{A g k K}{k + K}}.
\end{equation}
Duff \etal~\cite{Duff1962} generalized the linear theory to miscible interfaces and incorporated Chandrasekhar and Hide's viscous theories, producing an combined expression for the growth rate:
\begin{equation} \elabel{duff_simple}
\gamma = \sqrt{\frac{A g k}{\psi(A,k\delta)} + \nu^2 k^4} - (\nu + D) k^2,
\end{equation}
where 
$\delta$ is the instantaneous interface thickness,
$D$ is the diffusivity,
and $\psi$ is a function of the Atwood number and the product of the wavenumber and the interface thickness.
For $k \delta << 1$ and $A << 1$, $\psi \approx 1 + k \delta / \sqrt{\pi}$.
Note that for $D > 0$ there is a wavenumber cutoff above which the growth rate is negative, i.e., the perturbation decays.

In the uniform constant property case, $\delta = 2 \sqrt{D t}$, introducing a time-dependence on the linear stability:
\begin{equation} \elabel{duff_growth}
\gamma = \sqrt{\frac{A g k}{1 + \frac{2 k}{\sqrt{\pi}}\sqrt{D (t+t_0)} } + \nu^2 k^4} - (\nu + D) k^2,
\end{equation}
where $t_0$ is defined by the initial interface thickness:
\begin{equation}
t_0 = \frac{\delta_0^2}{4 D},
\end{equation}
where $\delta_0$ is the initial interface thickness.

\subsection{Weakly nonlinear expansions}
Jacobs and Catton provide a third order weakly non-linear theory for the inviscid unit Atwood Rayleigh-Taylor instability~\cite{Jacobs1988}.
Their weakly non-linear theory is primarily used to compare linear growth rates across a variety of perturbation symmetries in 3D.
Hexagonal and axi-symmetric perturbations are found to grow faster than rectangular perturbations.

Berning and Rubenchik extend the theory to arbitrary Atwood immiscible flows at any higher order, but analyze only the third order expansion~\cite{Berning1998}.
They perform a similar geometric comparison to Jacobs and Catton, but also use the harmonic couplings to characterize linear saturation.

The perturbation expansion has been taken to at least the 10th order by Liu \etal~\cite{Wang2010}.
However, there is limited progress to be made with such expansions, as singularities with branching point structures develop at moderate bubble displacements~\cite{Berning1998}.
Put another way, the interface and velocity potentials are not analytic in the span-wise position, e.g., when the interface rolls up.

\section{Potential flow models}

The next class of models to be applied to the Rayleigh-Taylor instability are potential flow models.
These models assume that little vorticity is generated and that it is confined to the interface, which is true at high Atwood numbers.
At moderate and low Atwood numbers, there is significant generation and transport of vorticity via, for example, the Kelvin-Helmholtz instability, so these models break down.

\subsection{Layzer's unit Atwood model}

One of the first potential flow models is due to Layzer~\cite{Layzer1955}.
Layzer's model is of an bubble with $\rho = 0$ rising in a fluid of density $\rho = 1$, which is unit Atwood number.
The bubble and fluid are assumed to be incompressible and inviscid.
The flow begins at rest, so there is no initial vorticity.
Layzer claims the flow will therefore continue to be irrotational, because the viscous generation term of the vorticity equation is zeroed for inviscid, incompressible flows.

Since the flow is inviscid and irrotational, Layzer uses the potential flow technique, writing the velocity as the gradient of a scalar potential:
\begin{equation}
v = \nabla \Phi,
\end{equation}
where 
$v$ is the velocity and 
$\Phi$ is the scalar potential.
Incompressibility zeroes the Laplacian of the potential:
\begin{equation}
\nabla^2 \Phi = 0.
\end{equation}
A Bernoulli equation is used model the interface:
\begin{equation} \label{eqn:LayzerBernoulli}
\begin{aligned}
f(t) & = \der{\Phi}{t}(\eta(r,t), r, t) \\
& - \frac{1}{2} \left(\left(\pder{\Phi}{z}\right)^2(\eta(r,t), r, t) +\left(\pder{\Phi}{r}\right)^2(\eta(r,t), r, t)\right) - g \eta(r,t),
\end{aligned}
\end{equation}
where 
$\eta(r,t)$ is the height of the interface,
$g$ is the gravitational acceleration, and 
$f(t)$ is an arbitrary function of time but not space.
The flow is axially symmetric with a vanishing radial component at transverse walls and vanishing vertical component far away from the bubble:
\begin{equation}
\pder{\Phi}{r}(z,R,t) = 0, \qquad \pder{\Phi}{z}(\pm \infty, r, t) = 0.
\end{equation}
Finally, the fluid advects the interface:
\begin{equation}
\der{\eta}{t}(r,t) = \pder{\Phi}{z}(\eta(r,t), r, t) - \pder{\Phi}{r}(\eta(r,t), r, t) \der{\eta}{r}(r,t),
\end{equation}

The bubble accelerates to a terminal velocity.
That velocity, in two and three dimensions, is:
\begin{equation}
V_{2d} = \frac{1}{\sqrt{3}} \sqrt{\frac{g R}{\pi}}, \qquad V_{cyl} = \sqrt{\frac{g R}{\beta_1}},
\end{equation}
where $\beta_1$ is the first root of the first order Bessel function of the first kind: $J_{1}(\beta_1) = 0$.
This velocity agrees with experimental results that were available to Layzer, e.g.\ those by Davies and Taylor~\cite{Davies1950a}.

\subsection{Goncharov's high Atwood model}

Goncharaov extends the Layzer model to include two fluids of arbitrary density difference.
In doing so, he makes a different choice of simplifying approximation for the Bernoulli equation~\cite{Goncharov2002}.
The consideration of a second fluid with non-zero density turns the Bernoulli equation \eref{LayzerBernoulli} into a difference:
\begin{equation} \label{eqn:GonBernoulli}
\begin{aligned}
f(t) = &   \rho_1 \der{\Phi_1}{t}(\eta(r,t), r, t) - \rho_2 \der{\Phi_2}{t}(\eta(r,t), r, t) \\
& - \rho_1 \frac{1}{2} \left(\left(\pder{\Phi_1}{z}\right)^2(\eta(r,t), r, t) +\left(\pder{\Phi_1}{r}\right)^2(\eta(r,t), r, t)\right) \\
& + \rho_2 \frac{1}{2} \left(\left(\pder{\Phi_2}{z}\right)^2(\eta(r,t), r, t) +\left(\pder{\Phi_2}{r}\right)^2(\eta(r,t), r, t)\right) \\
& - g \rho_1 \eta(r,t) + g \rho_2 \eta(r,t).
\end{aligned}
\end{equation}
The Goncharaov model keeps the free-slip boundary condition between the two fluids, which is exact only for $A = 1$ and a reasonable approximation for $\rho_1 / \rho_2 >> 1$.
In this respect, Goncharaov's model should be reasonable for high-Atwood, nearly inviscid flows.
The terminal velocity predicted is:
\begin{equation}
V = 1.02 \sqrt{\frac{2A }{1 + A} \frac{g}{k}} = \frac{1.02}{\sqrt{\pi}} \sqrt{\frac{A g \lambda}{1 + A}}.
\end{equation}
Similar potential flow models were introduced by Sohn~\cite{Sohn2003} and Abarzhi \etal\cite{Abarzhi2003}, with similar results.

\section{Buoyancy-drag models}

Buoyancy-drag models were developed concurrently with potential flow models, in part to provide a physical interpretation for their results.
They balance buoyant and parasitic forces related to the geometry of a model bubble.
Historically, buoyancy-drag models have had only 1 or 2 adjustable parameters, so they are evaluated more on their ability to reproduce specific features of the flow, e.g., the terminal velocity, rather than the full time-history.
Here, we focus on models applicable to single-mode non-interacting bubbles.

\subsection{Bubble model of Davies and Taylor}

Early experiments on the Rayleigh-Taylor instability by Davies and Taylor~\cite{Davies1950a} were performed by measuring the dynamics of large bubbles of gas rising through a dense liquid.
In their analysis, they relate the terminal velocity of the bubble to a drag coefficient, implicitly defining a buoyancy-drag model of the form:
\begin{equation} \elabel{dtbd}
\dot{v} \rho \mathcal{V} = \rho g \mathcal{V} - C_D \pi d^2 \frac{1}{2} \rho v^2,
\end{equation}
where $v$ is the gas bubble velocity,
$\rho$ is the density of the liquid,
$g$ is the gravitational acceleration,
$\mathcal{V}$ is the bubble volume,
$C_d$ is a drag coefficient, and
$d$ is the bubble diameter.
The coefficient $C_d$ was found to take values between $0.52$ and $1.37$.

\subsection{Tube model of Dimonte and Schneider}

Dimonte and Schneider develop a buoyancy-drag model for tube-shaped bubbles~\cite{Dimonte1996,Dimonte2000a} based on Davies and Taylor's model, \eref{dtbd}.
They let the ratio of the area to the volume go with the inverse bubble height, $\mathcal{A} / \mathcal{V} \sim 1/h$.
They also add a rescaling of the buoyant term by $\beta$, attributed to Youngs:
\begin{equation}
\dot{v_b}  = \beta A g - C_d \frac{v_b^2}{h_b}, 
\end{equation}
where $v_b$ is the bubble velocity,
$\beta < 1$ accounts for the relatively smaller buoyant portion of the bubble due to entrainment,
$C_d$ is a drag coefficient, and
$h_b$ is the bubble height.
$\beta$ and $C_d$ depend on the Atwood number, but Dimonte proposes $\beta = 1/2$ and $C_d = 2$ for $A << 1$~\cite{Dimonte2000}.
However, the model is stated to apply to self-similar bubble fronts, in which $h_b \sim D$.

\subsection{Self-similar model of Oron}

A model by Oron \etal also rescales the bubble mass~~\cite{Oron2001}:
\begin{equation} \elabel{buoyancy_drag}
(\rho_1 + C_a \rho_2) \ddot{h} = (\rho_2 - \rho_1)g - \frac{C_d}{\lambda} \dot{h}^2 \rho_2 \mathcal{A},
\end{equation}
where $\rho_2 > \rho_1$ are the densities of the two fluids, 
$C_a$ is an added mass coefficient,
$h$ is the height of the bubble,
$g$ is the gravitational acceleration,
$C_d$ is a drag-like coefficient, and
$\lambda$ is a characteristic length.
The use of $\lambda$, which is time-independent, implies the model is directed at self-similar flow.
The values of $C_a$ and $C_d$ are assumed to be Atwood independent and set to agree with Layzer's theory:
\begin{equation}
C_a = 1, \qquad C_d = 2\pi.
\end{equation}

\section{Problems with single mode Rayleigh-Taylor modeling}

Simulations by Ramaprabhu \etal ~\cite{Ramaprabhu2006} have shown that, after stagnating at a constant velocity in agreement with the potential flow models, low-Atwood bubbles re-accelerate to velocities nearly twice the potential flow limit.
The stagnation and re-acceleration phenomena were confirmed experimentally by Wilkinson and Jacobs~\cite{Wilkinson2007}.
Modeling the stagnation and re-acceleration phases is the primary open problem in the low Atwood single-mode Rayleigh-Taylor instability.
The desire to describe this re-acceleration process, quantitatively, is the motivation for this thesis.

First, though, I will give three qualitative explanations for why re-acceleration should have been expected: one based on the pressure balance, one based on the assumptions of potential flow models, and another by identifying a historical inconsistency in the development of buoyancy-drag models.

\subsection{Pressure in the single-mode RTI}

If there is a terminal velocity regime, can it be due to form drag?
In other words, can we have terminal velocity without viscosity?
Let $\nu \rightarrow 0$ and consider a fluid element lying on the axis
of a bubble or spike at terminal velocity.
By symmetry, it will only have a z-component of the velocity.
The z-forces must balance:
\begin{equation}
- \phi g \hat{z} = \pder{P}{z} ,
\end{equation}
where $\phi$ represents the mass.
In the falling spike, the pressure would be decreasing with $z$.
In the rising bubble, the pressure would be increasing with $z$.
There would necessarily be a pressure gradient between the head of the bubble and the tail of the spike, and vice versa.
As the bubble aspect ratio exceeded unity, the span-wise pressure gradient would exceed the gravitational forcing.
The resulting span-wise flow would rapidly mix the two fluids, destroying the bubble and spike.

The form of the bubble and spike require the pressure to be reasonably homogeneous span-wise.
Only at the bubble and spike tips do we observe a span-wise flow: the displacement of stationary fluid by the tip.
In other words, the pressure drag is highly localized to the bubble and spike tips but cannot affect the flow in the stems of the bubbles and spikes.
If the flow is terminal, then it must be attenuated predominately by viscous drag, which can act along the sidewalls, that is the stem, of the bubbles and spikes.

\subsection{Departure from potential flow}
The assumption that the flow is irrotational applies only at high Atwood number.
At moderate and low Atwood numbers, the interface between the light and the dense fluid is a shear layer that generates vorticity.
If the viscosity is low enough, secondary Kelvin-Helmholtz instabilities develop in the shear layer and transport vorticity into the center of the bubble.
While it is not obvious that vorticity should cause re-acceleration, it is clear that the flow is not irrotational, even away from the fluid interface, and therefore cannot be accurately modeled by potential flow.

\subsection{Historical inconsistency in buoyancy-drag models}

Buoyancy-drag models contain a buoyant term that goes with the bubble's volume and a form drag term that goes with the bubble's span-wise area.
The model was originally developed to describe multi-mode self-similar flow, in which there is only one length scale, the dominant wavelength $\lambda$.
Consequently, the ratio of the volume to the surface area is $\lambda^{-1}$, yielding a terminal velocity as a function of $\lambda$.

However, the single-mode RTI has two length scales: in addition to the wavelength $\lambda$ there is the bubble height, $h$.
In other words, single-mode RTI bubbles are cylindrical instead of spherical with an axis length that goes with the bubble height.
The ratio of the volume to surface area is $h^{-1}$, not $\lambda^{-1}$, so force balance occurs when $\dot{h} \sim \sqrt{h}$, which is not terminal.
Only by introducing a drag term that goes with the height $h$, such as skin drag, can a terminal velocity be recovered.
This terminal velocity would be a function of the viscosity, and therefore cannot be described by potential flow.

\section{Related work aimed at modeling stagnation and re-acceleration}
Since re-acceleration was observed experimentally, multiple attempts have been made to capture re-acceleration in the models.

\subsection{Vortex ring correction of Ramaprabhu}

Ramaprabhu \etal ~\cite{Ramaprabhu2012} attribute the reacceleration to the formation of a vortex ring at the bubble tip.
They add a term to their buoyancy-drag model representing the centrifugal force per unit volume:
\begin{equation}
\left(\rho_2 g - \rho_1 g\right) + \rho_1 \frac{\omega_0^2 R}{ 4} = \frac{C_d \rho_2 v^2}{\lambda},
\end{equation}
where $\omega_0$ is the average vorticity in the bubble tip.
The model does not provide an evolution equation for $\omega_0$; it is measured from simulations ad-hoc making the model descriptive but not predictive.
The model agrees qualitatively, but not quantitatively, from the onset of stagnation at bubble height $h / \lambda \approx 0.5$ through re-acceleration at $h/\lambda \approx 2.0$, but doesn't capture linear growth at early times or the dynamics at late times.
Furthermore, they compare the vortex ring model to simulations using two different codes; the two codes disagree quantitatively over the re-acceleration regime and disagree qualitatively over what follows it.

\section{Vorticity and viscosity in potential flow}

Banerjee \etal attempt to describe re-acceleration by adding viscous and vortical effects to a potential flow model~\cite{Banerjee2011}.
Similar to the vortex ring correction to buoyancy-drag, the vorticity is an input to the potential flow model.
Instead of using data from simulations, Banerjee \etal write the vorticity as an analytic function of time independent of the Atwood number.
The resulting dynamics have a single re-acceleration phase before reaching an asymptotic terminal velocity.

 % Introduction

%\chapter{Related work}

The shortcomings of models for the single mode low Atwood Rayleigh-Taylor instability, specifically re-acceleration, were first recognized in 2006 by Ramaprabhu et al.~\cite{Ramaprabhu2006} based on numerical simulations.
Experimental confirmation by Wilkinson and Jacobs followed a year later~\cite{Wilkinson2007}.
Since then, multiple attempts have been made to capture re-acceleration in the models.
In this chapter, we review those attempts.

\section{Vortex ring correction of Ramaprabhu}

Ramaprabhu et al. attribute the reacceleration to the formation of a vortex ring at the bubble tip.
They add a term to their buoyancy-drag model representing the centrifugal force per unit volume:
\begin{equation}
\left(\rho_2 g - \rho_1 g\right) + \rho_1 \frac{\omega_0^2 R}{ 4} = \frac{C_d \rho_2 v^2}{\lambda}
\end{equation}
where $\omega_0$ is the average vorticity in the bubble tip.
The model does not provide an evolution equation for $\omega_0$; it is measured from simulations ad-hoc making the model descriptive but not predictive.

 % Introduction

\chapter{Standalone papers}

This thesis contains three standalone papers.
They are logically ordered from the top down.

The first paper contains a new buoyancy-drag model developed to describe the late time behavior in the low-Atwood, moderate Grashof case.
It contains model coefficients that are fit to a data set of direct numerical simulations.
This single author paper will be submitted to for peer review.

The second paper validates those direct numerical simulations against experimental data from Wilkinson and Jacobs~\cite{Wilkinson2007}.
This establishes that the simulations contain the physical processes responsible for re-acceleration and other unexplained late-time phenomena.
The paper takes advantage of the generality of numerical data to explore feature of the flow that were not available to the original experiment, namely the interaction between the bubbles and pressure driving secondary flows in the mid-plane.
This single author paper has been submitted to peer review.

The third paper contains a performance and convergence study of the numerical method and simulation software with the single mode Rayleigh-Taylor problem as a benchmark.
The NekBox code, which was specialized specifically to this project, is shown to be an efficient tool for performing these calculations.
Furthermore, the resolution is selected such that the simulation error is an order smaller than the expected model error, which ensures that the flow is sufficiently but not over resolved.
Maxwell Hutchinson authored all of sections 2 and 4 and the majority of sections 1 and 5.
This paper has been peer reviewed and published in the proceedings of the International Conference on High Performance Computing (ISC).



%\input{Chapters/SingleModeLowRe} % Background Theory 

%\input{Chapters/SingleModeHighRe} % Experimental Setup


%\input{Chapters/SchmidtProjection} % Experiment 1

%\input{Chapters/Wilkinson} % Experiment 2

%\input{Chapters/Implications} % Implications

\chapter{Conclusions}

\section{New model for low-Atwood single mode}

\section{Validation of DNS}

\section{Importance of Schmidt number}

\section{Open questions}

 % Conclusion

%\input{Chapters/Chapter6} % Results and Discussion


%% ----------------------------------------------------------------
% Now begin the Appendices, including them as separate files

\addtocontents{toc}{\vspace{2em}} % Add a gap in the Contents, for aesthetics

\appendix % Cue to tell LaTeX that the following 'chapters' are Appendices

\chapter{Derivations}

\section{Linear stability theory} \alabel{LST}
We start with the incompressible Euler equation:
\begin{equation}
	\rho \left[ \frac{\partial u}{\partial t} + u \nabla u\right] = -\nabla P + g \rho \qquad \nabla \cdot u = 0 
\end{equation}

This is a linear theory, so we expand the pressure and density about their mean values, $P = \bar{P} + \tilde{P}$ and $\rho = \bar\rho + \tilde\rho$:
\begin{equation}
(\bar\rho +\tilde \rho) \left[ \frac{\partial u}{\partial t} + u \nabla u\right] = -\nabla \tilde{P} + g (\bar\rho+\tilde\rho) \qquad \nabla \cdot u = 0,
\end{equation}
and assume $\tilde{P}$ and $\vec{u}$ are small:
\begin{equation}
\bar\rho \frac{\partial u}{\partial t} = -\nabla \tilde{P} + g \tilde\rho \qquad \nabla \cdot u = 0
\end{equation}

We do a spectral decomposition of the velocity and pressure:
\begin{equation}
F(x,z,t) \sim \exp(i k x) \exp(i \gamma t) f(z),
\end{equation}
where $F$ is either $u$ or $P$, $k$ is the wavenumber, $k = 2 \pi / \lambda$, and $\gamma$ is the growth rate.

We separate the momentum equation into its components, $\vec{u} = (u,w)$, and substitute the spectral decomposition:
\begin{align}
	\bar\rho \gamma u &= - k \tilde{P} \elabel{span_p} \\
	i \bar\rho \gamma w &= - \frac{d\tilde{P}}{dz} - g \tilde\rho \elabel{virt_p}
\end{align}
Doing the same for incompressibility:
\begin{align}
	0 &= i k u + \frac{dw}{dz} \elabel{span_r} \\
	0 &= i \gamma \tilde\rho + w \frac{d\bar\rho}{dz} \elabel{virt_r}
\end{align}

We combine \eref{span_p} and \eref{span_r}:
\begin{equation}
\gamma \bar\rho \frac{dw}{dz} = i k^2 \tilde{P},
\end{equation}
and take the derivative in the $z$ direction:
\begin{equation}
\gamma \frac{d}{dz} \left(\bar\rho \frac{dw}{dz}\right) = i k^2 \frac{d\tilde{P}}{dz} .
\end{equation}
Next, we substitute in \eref{virt_p}:
\begin{equation}
\gamma \frac{d}{dz} \left(\bar\rho \frac{dw}{dz}\right) = - i k^2 \left(i \bar\rho \gamma w + g \tilde\rho\right) ,
\end{equation}
and then \eref{virt_r}:
\begin{equation} \elabel{foobar}
\gamma \frac{d}{dz} \left(\bar\rho \frac{dw}{dz}\right) = - i k^2 \left(i \bar\rho \gamma w + \frac{i g w}{\gamma} \frac{d \bar\rho}{dz}\right) .
\end{equation}
Before the next step, we simplify \eref{foobar}:
\begin{equation} \elabel{barfoo}
\frac{d}{dz} \left(\bar\rho \frac{dw}{dz}\right) = w k^2 \left(\bar\rho   + \frac{ g }{\gamma^2} \frac{d \bar\rho}{dz}\right)
\end{equation}

The interface is sharp, so, away from the interface, $\bar\rho$ is constant and \eref{barfoo} becomes:
\begin{equation}
\frac{d^2}{dz^2}w  = w k^2.
\end{equation}
Therefore, in the upper fluid $f(z) = \exp(-kz)$ and the velocity is, up to a constant:
\begin{equation}\elabel{w1}
w(z>0) \sim \exp(ikx) \exp(i\gamma t) \exp(-kz),
\end{equation}
while, in the lower fuild, the z dependence is reversed:
\begin{equation}\elabel{w2}
w(z\le0) \sim \exp(ikx) \exp(i\gamma t) \exp(kz),
\end{equation}


Now integrate across the interface from $z= -\Delta z/2$ to $z = \Delta z/2$:
\begin{equation}
\left(\bar\rho \frac{dw}{dz}\right)_2 - \left(\bar\rho \frac{dw}{dz}\right)_1 = w k^2 \bar\rho \Delta z  + \left(w k^2 \frac{ g }{\gamma^2}  \bar\rho\right)_2 - \left(w k^2 \frac{ g }{\gamma^2}  \bar\rho\right)_1 .
\end{equation}
From \eref{w1} and \eref{w2}, we know $dw/dz \sim \pm kw$.
Substituting that and letting $\Delta z \rightarrow 0$:
\begin{equation}
\bar\rho_2 k w + \bar\rho_1 k w = w k^2 \frac{ g }{\gamma^2}  (\bar\rho_2 - \bar\rho_1).
\end{equation}
The square growth rate is therefore:
\begin{equation}
\gamma^2 = k g \frac{\bar\rho_2 - \bar\rho_1}{\bar\rho_2+\bar\rho_1}  \equiv A g k.
\end{equation}


\section{Layzer model} \alabel{Layzer}

The Layzer potential flow model assumes the flow is invicid and irrotational~\cite{Layzer1955}.
In this case, the flow velocity can be written as the gradient of a scalar velocity potential:
\begin{equation}
v = \nabla \phi
\end{equation}
where 
$v$ is the velocity and 
$\phi$ is the scalar potential.
In the Layzer model, the flow is additionally incompressible, in which case the Laplacian of the potential is zero:
\begin{equation}
\nabla^2 \phi = 0
\end{equation}
Finally, the Layzer model has unit Atwood number, so there is only a single velocity potential corresponding to the flow of the heavy fluid.

The governing equation is a Bernoulli equation at the interface between the modeled heavy fluid and light bubble:
\begin{equation} \elabel{layzer_jump}
\der{\phi}{t}(r, \eta(r,t), t) + \frac{1}{2} \left(\left(\pder{\phi}{z}\right)^2(r, \eta(r,t), t) +\left(\pder{\phi}{r}\right)^2(r, \eta(r,t), t)\right) + g \eta(r,t) = f(t)
\end{equation}
where 
$\eta(r,t)$ is the height of the interface,
$g$ is the gravitational acceleration, and 
$f(t)$ is an arbitrary function of time but not space.
The flow is axially symmetric with a vanishing radial component at transverse walls and vanishing vertical component far away from the bubble:
\begin{equation}
\pder{\phi}{r}(R,z,t) = 0 \qquad \pder{\phi}{z}(r, \pm \infty, t) = 0 .
\end{equation}
Finally, the fluid advects the interface:
\begin{equation} \elabel{layzer_adv}
\der{\eta}{t}(r,t) = \pder{\phi}{z}(r, \eta(r,t), t) - \pder{\phi}{r}(r, \eta(r,t), t) \der{\eta}{r}(r,t).
\end{equation}

The simplest non-trivial potential that fits the boundary conditions is:
\begin{equation}
\phi(r,z,t) = F(t) J_0(r) e^{-z}.
\end{equation}

We substitute this potential into \eref{layzer_jump} and \eref{layzer_adv}:
\begin{align}
f(t) &= F'(t) J_0(r) e^{-\eta(r,t)} + \frac{F^2(t) e^{-2\eta(r,t)}}{2} \left[ J_0^2(r)  + J_1^2(r) \right] + g \eta(r,t) ,\\
\der{\eta}{t}(r,t) &=  -F(t) J_0(r)e^{-\eta(r,t)}  + F(t) J_1(r) e^{-\eta(r,t)} \der{\eta}{r}(r,t)
\end{align}

These equations cannot be satisifed for all $r$, given the form of the ansatz.
Layzer decided to satisfy the Bernoulli and advection equations in the neighborhood of the bubble tip by expanding $\eta(r,t) \approx \eta_0(t) + r^2 \eta_2(t)$.
We substitute in the quadratic bubble tip model:
\begin{align}
f(t) &= F'(t) J_0(r) e^{-\eta_0(t)-\eta_2(t)r^2} \\ \nonumber
	&+ \frac{F^2(t) e^{-2\eta_0(t)-2\eta_2(t)r^2}}{2} \left[ J_0^2(r)  + J_1^2(r) \right] + g \eta_0(t) + g \eta_2(t)r^2 ,\\
\dot\eta_0 + r^2 \dot\eta_2 &=  -F(t) J_0(r)e^{-\eta_0(t) - \eta_2 r^2}  + 2 F(t) J_1(r) e^{-\eta_0(t) - \eta_2(t)r^2} r \eta_2(t) .
\end{align}

We discard terms smaller than $r^2$:
\begin{align}
f(t) &= F'(t) \left(1-r^2/4\right) (1-\eta_2(t)r^2) e^{-\eta_0(t)} \\ \nonumber
	&+ \frac{F^2(t) (1-2\eta_2(t)r^2) e^{-2\eta_0(t)}}{2} \left[1-r^2/4\right] + g \eta_0(t) + g \eta_2(t)r^2 ,\\
\dot\eta_0 + r^2 \dot\eta_2 &= (-1+r^2/4 + r^2 \eta_2(t)) F(t) (1-\eta_2(t)r^2) e^{-\eta_0(t)} .
\end{align}

Next, we collect the terms based on their $r$-dependence:
\begin{align}
f(t) &= F'(t) e^{-\eta_0(t)} + \frac{F^2(t) e^{-2\eta_0(t)}}{2} + g \eta_0(t) , \\
0 &= - F'(t)(\eta_2(t)+ 1/4) e^{-\eta_0(t)} - \frac{F^2(t) e^{-2\eta_0(t)}}{2}(2\eta_2(t)+1/4) + g \eta_2(t) ,\\
\dot\eta_0  &= - F(t)  e^{-\eta_0(t)} , \\
\dot\eta_2 &= (1/4 + 2\eta_2(t)) F(t)  e^{-\eta_0(t)} ,
\end{align}
and simplify:
\begin{align}
f(t) &= F'(t) e^{-\eta_0(t)} + \frac{F^2(t) e^{-2\eta_0(t)}}{2} + g \eta_0(t)  \\
0 &= - F'(t) e^{-\eta_0(t)} (4\eta_2(t)+ 1)- \frac{F^2(t) e^{-2\eta_0(t)}}{2}(8\eta_2(t)+1) + 4g \eta_2(t) \\
\dot\eta_0  &= - F(t)  e^{-\eta_0(t)}  \\
\dot\eta_2 &= (1/4 + 2\eta_2(t)) F(t)  e^{-\eta_0(t)} .
\end{align}

We expect the bubble curvature to stabalize at late times, so $\eta_2 = -1/8$ such that $\dot\eta_2 = 0$.
We substitute this in:
\begin{align}
f(t) &= F'(t) e^{-\eta_0(t)} + \frac{F^2(t) e^{-2\eta_0(t)}}{2} + g \eta_0(t) , \\
0 &=  F'(t) e^{-\eta_0(t)}  + g , \\
	\dot\eta_0  &= - F(t)  e^{-\eta_0(t)} . \elabel{etaev}
\end{align}

We can use \eref{etaev} to eliminate $F$:
\begin{equation}
0 = -\left(\ddot\eta_0(t)+(\dot\eta_0(t))^2\right) + g 
\end{equation}

The solution to this second order nonlinear ordinary differential equation is:
\begin{equation}
\eta_0(t) = \log\left( \exp\left[2 \sqrt{g} (c_1 + t) \right] + 1 \right) - \sqrt{g}t + c_2.
\end{equation}
We are interested in the asymptotic velocity, i.e.\ $t >> 1$:
\begin{equation}
\eta_0(t>>1) \approx  \sqrt{g}t .
\end{equation}

Finally, we substitute back $r = \frac{R}{\beta_1}$, where $J_1(\beta_1) = 0$:
\begin{equation}
V = \sqrt{\frac{g R}{\beta_1}} \approx 0.511 \sqrt{g R}
\end{equation}

	% Appendix Title

%\chapter{Observables}

\section{Incompressible flows}

\subsection{Energy balance}

The kinetic energy of a fluid flow is defined as:
\begin{equation} \elabel{kinetic-energy}
T = \int_\Omega \left[\frac{1}{2} \rho u^2 \right] dV
\end{equation}

The gravitational potential is:
\begin{equation} \elabel{grav-energy}
V_g = \int_\Omega \left[ \rho  g \cdot r \right] dV
\end{equation}

The rate of local dissipation:
\begin{equation} \elabel{diss-vol-rate}
\Phi = \nu \left[ 2 \left(\pder{u_i}{x_i}\right)^2 + \left(u_i u_j \epsilon_{i,j,k}\right) \left(u_I u_J \epsilon_{I,J,k}\right) \right]
\end{equation}
 % Appendix Title

%\chapter{Adjoints}

\section{Example: Burger's equation}

Let the objective function be the total kinetic energy, $T = \frac{1}{2} \int u^2 dx$.
This can be written as:
\begin{align} \elabel{burg_T}
T &= \frac{1}{2} \int u(x,t)^2 dx \\
&= \int dx \\
&\left[-2 \nu \pder{}{x} \ln \left\{(4 \pi \nu t)^{-1/2} \int_{-\infty}^\infty \exp\left[-\frac{(x-x')^2}{4\nu t} - \frac{1}{2\nu} \int_0^{x'} u(x'',0) dx''\right] dx'\right\} \right]^2 \\
\end{align}

We use the chain rule:
\begin{equation}
\pder{T(t)}{u(x,0)} = \pder{T}{u(y,t)} \pder{u(y,t)}{\phi(z,t)} \pder{\phi(z,t)}{\phi(w,0)} \pder{\phi(w,0)}{u(x,0} 
\end{equation}
We can write each of these down:
\begin{align*}
\pder{T}{u(y,t)} &= u(y,t) \\
\pder{u(y,t)}{\phi(z,t)} &= -\frac{2 \nu}{\phi(z,t)} \left[\ppder{\phi(z,t)}{z} - \frac{1}{\phi} \pder{\phi(z,t)}{z}\right] \delta(y-z) \\
\pder{\phi(z,t)}{\phi(w,0)} &= \frac{1}{\sqrt{4 \nu \pi t}} e^{-(z-w)^2/(4 \nu t)} \\
\pder{\phi(w,0)}{u(x,0)} &= - \frac{\phi(x,0)}{2 \nu} \Theta(w-x) 
\end{align*}
 % Appendix Title

\addtocontents{toc}{\vspace{2em}}  % Add a gap in the Contents, for aesthetics
\backmatter

%% ----------------------------------------------------------------
\label{Bibliography}
\lhead{\emph{Bibliography}}  % Change the left side page header to "Bibliography"
\bibliographystyle{unsrtnat}  % Use the "unsrtnat" BibTeX style for formatting the Bibliography
\bibliography{library}  % The references (bibliography) information are stored in the file named "Bibliography.bib"

\end{document}  % The End
%% ----------------------------------------------------------------
