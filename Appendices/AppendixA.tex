\chapter{Derivations}

\section{Linear stability theory} \alabel{LST}
We start with the incompressible Euler equation:
\begin{equation}
	\rho \left[ \frac{\partial u}{\partial t} + u \nabla u\right] = -\nabla P + g \rho \qquad \nabla \cdot u = 0 
\end{equation}

This is a linear theory, so we expand the pressure and density about their mean values, $P = \bar{P} + \tilde{P}$ and $\rho = \bar\rho + \tilde\rho$:
\begin{equation}
(\bar\rho +\tilde \rho) \left[ \frac{\partial u}{\partial t} + u \nabla u\right] = -\nabla \tilde{P} + g (\bar\rho+\tilde\rho) \qquad \nabla \cdot u = 0,
\end{equation}
and assume $\tilde{P}$ and $\vec{u}$ are small:
\begin{equation}
\bar\rho \frac{\partial u}{\partial t} = -\nabla \tilde{P} + g \tilde\rho \qquad \nabla \cdot u = 0
\end{equation}

We do a spectral decomposition of the velocity and pressure:
\begin{equation}
F(x,z,t) \sim \exp(i k x) \exp(i \gamma t) f(z),
\end{equation}
where $F$ is either $u$ or $P$, $k$ is the wavenumber, $k = 2 \pi / \lambda$, and $\gamma$ is the growth rate.

We separate the momentum equation into its components, $\vec{u} = (u,w)$, and substitute the spectral decomposition:
\begin{align}
	\bar\rho \gamma u &= - k \tilde{P} \elabel{span_p} \\
	i \bar\rho \gamma w &= - \frac{d\tilde{P}}{dz} - g \tilde\rho \elabel{virt_p}
\end{align}
Doing the same for incompressibility:
\begin{align}
	0 &= i k u + \frac{dw}{dz} \elabel{span_r} \\
	0 &= i \gamma \tilde\rho + w \frac{d\bar\rho}{dz} \elabel{virt_r}
\end{align}

We combine \eref{span_p} and \eref{span_r}:
\begin{equation}
\gamma \bar\rho \frac{dw}{dz} = i k^2 \tilde{P},
\end{equation}
and take the derivative in the $z$ direction:
\begin{equation}
\gamma \frac{d}{dz} \left(\bar\rho \frac{dw}{dz}\right) = i k^2 \frac{d\tilde{P}}{dz} .
\end{equation}
Next, we substitute in \eref{virt_p}:
\begin{equation}
\gamma \frac{d}{dz} \left(\bar\rho \frac{dw}{dz}\right) = - i k^2 \left(i \bar\rho \gamma w + g \tilde\rho\right) ,
\end{equation}
and then \eref{virt_r}:
\begin{equation} \elabel{foobar}
\gamma \frac{d}{dz} \left(\bar\rho \frac{dw}{dz}\right) = - i k^2 \left(i \bar\rho \gamma w + \frac{i g w}{\gamma} \frac{d \bar\rho}{dz}\right) .
\end{equation}
Before the next step, we simplify \eref{foobar}:
\begin{equation} \elabel{barfoo}
\frac{d}{dz} \left(\bar\rho \frac{dw}{dz}\right) = w k^2 \left(\bar\rho   + \frac{ g }{\gamma^2} \frac{d \bar\rho}{dz}\right)
\end{equation}

The interface is sharp, so, away from the interface, $\bar\rho$ is constant and \eref{barfoo} becomes:
\begin{equation}
\frac{d^2}{dz^2}w  = w k^2.
\end{equation}
Therefore, in the upper fluid $f(z) = \exp(-kz)$ and the velocity is, up to a constant:
\begin{equation}\elabel{w1}
w(z>0) \sim \exp(ikx) \exp(i\gamma t) \exp(-kz),
\end{equation}
while, in the lower fuild, the z dependence is reversed:
\begin{equation}\elabel{w2}
w(z\le0) \sim \exp(ikx) \exp(i\gamma t) \exp(kz),
\end{equation}


Now integrate across the interface from $z= -\Delta z/2$ to $z = \Delta z/2$:
\begin{equation}
\left(\bar\rho \frac{dw}{dz}\right)_2 - \left(\bar\rho \frac{dw}{dz}\right)_1 = w k^2 \bar\rho \Delta z  + \left(w k^2 \frac{ g }{\gamma^2}  \bar\rho\right)_2 - \left(w k^2 \frac{ g }{\gamma^2}  \bar\rho\right)_1 .
\end{equation}
From \eref{w1} and \eref{w2}, we know $dw/dz \sim \pm kw$.
Substituting that and letting $\Delta z \rightarrow 0$:
\begin{equation}
\bar\rho_2 k w + \bar\rho_1 k w = w k^2 \frac{ g }{\gamma^2}  (\bar\rho_2 - \bar\rho_1).
\end{equation}
The square growth rate is therefore:
\begin{equation}
\gamma^2 = k g \frac{\bar\rho_2 - \bar\rho_1}{\bar\rho_2+\bar\rho_1}  \equiv A g k.
\end{equation}


\section{Layzer model} \alabel{Layzer}

The Layzer potential flow model assumes the flow is invicid and irrotational~\cite{Layzer1955}.
In this case, the flow velocity can be written as the gradient of a scalar velocity potential:
\begin{equation}
v = \nabla \phi
\end{equation}
where 
$v$ is the velocity and 
$\phi$ is the scalar potential.
In the Layzer model, the flow is additionally incompressible, in which case the Laplacian of the potential is zero:
\begin{equation}
\nabla^2 \phi = 0
\end{equation}
Finally, the Layzer model has unit Atwood number, so there is only a single velocity potential corresponding to the flow of the heavy fluid.

The governing equation is a Bernoulli equation at the interface between the modeled heavy fluid and light bubble:
\begin{equation} \elabel{layzer_jump}
\der{\phi}{t}(r, \eta(r,t), t) + \frac{1}{2} \left(\left(\pder{\phi}{z}\right)^2(r, \eta(r,t), t) +\left(\pder{\phi}{r}\right)^2(r, \eta(r,t), t)\right) + g \eta(r,t) = f(t)
\end{equation}
where 
$\eta(r,t)$ is the height of the interface,
$g$ is the gravitational acceleration, and 
$f(t)$ is an arbitrary function of time but not space.
The flow is axially symmetric with a vanishing radial component at transverse walls and vanishing vertical component far away from the bubble:
\begin{equation}
\pder{\phi}{r}(R,z,t) = 0 \qquad \pder{\phi}{z}(r, \pm \infty, t) = 0 .
\end{equation}
Finally, the fluid advects the interface:
\begin{equation} \elabel{layzer_adv}
\der{\eta}{t}(r,t) = \pder{\phi}{z}(r, \eta(r,t), t) - \pder{\phi}{r}(r, \eta(r,t), t) \der{\eta}{r}(r,t).
\end{equation}

The simplest non-trivial potential that fits the boundary conditions is:
\begin{equation}
\phi(r,z,t) = F(t) J_0(r) e^{-z}.
\end{equation}

We substitute this potential into \eref{layzer_jump} and \eref{layzer_adv}:
\begin{align}
f(t) &= F'(t) J_0(r) e^{-\eta(r,t)} + \frac{F^2(t) e^{-2\eta(r,t)}}{2} \left[ J_0^2(r)  + J_1^2(r) \right] + g \eta(r,t) ,\\
\der{\eta}{t}(r,t) &=  -F(t) J_0(r)e^{-\eta(r,t)}  + F(t) J_1(r) e^{-\eta(r,t)} \der{\eta}{r}(r,t)
\end{align}

These equations cannot be satisifed for all $r$, given the form of the ansatz.
Layzer decided to satisfy the Bernoulli and advection equations in the neighborhood of the bubble tip by expanding $\eta(r,t) \approx \eta_0(t) + r^2 \eta_2(t)$.
We substitute in the quadratic bubble tip model:
\begin{align}
f(t) &= F'(t) J_0(r) e^{-\eta_0(t)-\eta_2(t)r^2} \\ \nonumber
	&+ \frac{F^2(t) e^{-2\eta_0(t)-2\eta_2(t)r^2}}{2} \left[ J_0^2(r)  + J_1^2(r) \right] + g \eta_0(t) + g \eta_2(t)r^2 ,\\
\dot\eta_0 + r^2 \dot\eta_2 &=  -F(t) J_0(r)e^{-\eta_0(t) - \eta_2 r^2}  + 2 F(t) J_1(r) e^{-\eta_0(t) - \eta_2(t)r^2} r \eta_2(t) .
\end{align}

We discard terms smaller than $r^2$:
\begin{align}
f(t) &= F'(t) \left(1-r^2/4\right) (1-\eta_2(t)r^2) e^{-\eta_0(t)} \\ \nonumber
	&+ \frac{F^2(t) (1-2\eta_2(t)r^2) e^{-2\eta_0(t)}}{2} \left[1-r^2/4\right] + g \eta_0(t) + g \eta_2(t)r^2 ,\\
\dot\eta_0 + r^2 \dot\eta_2 &= (-1+r^2/4 + r^2 \eta_2(t)) F(t) (1-\eta_2(t)r^2) e^{-\eta_0(t)} .
\end{align}

Next, we collect the terms based on their $r$-dependence:
\begin{align}
f(t) &= F'(t) e^{-\eta_0(t)} + \frac{F^2(t) e^{-2\eta_0(t)}}{2} + g \eta_0(t) , \\
0 &= - F'(t)(\eta_2(t)+ 1/4) e^{-\eta_0(t)} - \frac{F^2(t) e^{-2\eta_0(t)}}{2}(2\eta_2(t)+1/4) + g \eta_2(t) ,\\
\dot\eta_0  &= - F(t)  e^{-\eta_0(t)} , \\
\dot\eta_2 &= (1/4 + 2\eta_2(t)) F(t)  e^{-\eta_0(t)} ,
\end{align}
and simplify:
\begin{align}
f(t) &= F'(t) e^{-\eta_0(t)} + \frac{F^2(t) e^{-2\eta_0(t)}}{2} + g \eta_0(t)  \\
0 &= - F'(t) e^{-\eta_0(t)} (4\eta_2(t)+ 1)- \frac{F^2(t) e^{-2\eta_0(t)}}{2}(8\eta_2(t)+1) + 4g \eta_2(t) \\
\dot\eta_0  &= - F(t)  e^{-\eta_0(t)}  \\
\dot\eta_2 &= (1/4 + 2\eta_2(t)) F(t)  e^{-\eta_0(t)} .
\end{align}

We expect the bubble curvature to stabalize at late times, so $\eta_2 = -1/8$ such that $\dot\eta_2 = 0$.
We substitute this in:
\begin{align}
f(t) &= F'(t) e^{-\eta_0(t)} + \frac{F^2(t) e^{-2\eta_0(t)}}{2} + g \eta_0(t) , \\
0 &=  F'(t) e^{-\eta_0(t)}  + g , \\
	\dot\eta_0  &= - F(t)  e^{-\eta_0(t)} . \elabel{etaev}
\end{align}

We can use \eref{etaev} to eliminate $F$:
\begin{equation}
0 = -\left(\ddot\eta_0(t)+(\dot\eta_0(t))^2\right) + g 
\end{equation}

The solution to this second order nonlinear ordinary differential equation is:
\begin{equation}
\eta_0(t) = \log\left( \exp\left[2 \sqrt{g} (c_1 + t) \right] + 1 \right) - \sqrt{g}t + c_2.
\end{equation}
We are interested in the asymptotic velocity, i.e.\ $t >> 1$:
\begin{equation}
\eta_0(t>>1) \approx  \sqrt{g}t .
\end{equation}

Finally, we substitute back $r = \frac{R}{\beta_1}$, where $J_1(\beta_1) = 0$:
\begin{equation}
V = \sqrt{\frac{g R}{\beta_1}} \approx 0.511 \sqrt{g R}
\end{equation}

