\section{Conclusions} \slabel{concs}

The simulations described here reproduce the growth rate, stagnation velocity, and re-acceleration of the low-Atwood single mode Rayleigh Taylor instability for three experimental runs by Wilkinson and Jacobs.
These reproductions inspire confidence not only in the NekBox code, but also in the Boussinesq approximation for $A = 0.15$ and the low-Schmidt approximation.

In wall-bounded flows, the bubbles and spikes nearest to the no-slip boundaries experience lift and drag forces that slow their non-linear growth and push them towards their inner neighbors.
There is an additional effect due to the finite domain breaking one of the 4-fold symmetries from the purely periodic problem.
In the wall-bounded initial condition, one corner of the domain has an excess of bubbles while the opposite has an excess of spikes.
This sets up a long-wavelength mode across the diagonal that encourages bubble growth in one corner and discourages it in the other.

Ultimately, the bubble-bubble and spike-spike collisions may destroy the single-mode ordering of the flow at aspect ratio 5, but the onset of velocity decay may alternatively be due to the upper boundary.
If the decay is due to collisions, it would limit the use of wall-bounded flows as proxies for periodic flows to moderate aspect ratios.
The ability of wall-bounded flows to approximate periodic ones at high aspect ratio warrants further study.

In addition to causing collisions, the growing boundary layer squeezes the flow in the span-wise direction, accelerating it past its fully-periodic trajectory in the stagnation and re-acceleration phases.

The inner bubbles experience near constant acceleration from aspect ratio 2 to aspect ratio 5, with a maximum Froude number of 1.8.
This contrasts results by Ramaprabhu et al.~\cite{Ramaprabhu2012} that show saturation post-reacceleration at $\text{Fr} \approx 1$.
The saturation could be explained by excess mixing or the finite size of the domain, which extends to $h/\lambda = 6$ in their case and $9$ in ours.
Alternatively, the acceleration could be artificially sustained by the wall lift force pushing the boundary bubbles into the interior ones.

Single-mode Rayleigh-Taylor flows develop span-wise pressure gradients with local minima in bubble and spike centers and local maxima in bubble and spike corners.
The pressure drives secondary flows of the first kind in the form of vortex quads centered on bubble-spike interface centers.
These span-wise flows mix the fluid across otherwise laminar interfaces, perturbing the scalar profiles.

\section{Acknowledgements}

M. H. is grateful for useful conversations with Jeffrey Jacobs, Robert Roser, Aleksandr Obabko, Elia Merzari, Oana Marin, and Elizabeth Hicks.

M. H. acknowledges support from the Department of Energy Computational Science graduate fellowship.
This research used resources of the Argonne Leadership Computing Facility, which is a DOE Office of Science User Facility supported under Contract DE-AC02-06CH11357.

