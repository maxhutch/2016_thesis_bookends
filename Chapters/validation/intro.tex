\section{Introduction}

The Rayleigh-Taylor instability occurs when a denser fluid is supported by a lighter one.
Low-amplitude perturbations in the interface between the two fluids grow exponentially with a rate that is well modeled by linear stability analysis~\cite{Duff1962}:
\begin{equation} \elabel{duff}
\gamma = \sqrt{\frac{Agk}{\psi} + \nu^2 k^4} - (\nu + D) k^2,
\end{equation}
where 
$g$ is the local acceleration,
$k$ is the wave-number,
$\nu$ is the kinematic viscosity,
$D$ is the diffusivity,
$A$ is the Atwood number, which characterizes the density difference:
\begin{equation}
	A = \frac{\rho_h - \rho_l}{\rho_h + \rho_l},
\end{equation}
and $\psi$ describes the effect of the interface thickness and is a function of $A$, $k$, and the thickness $\delta$.
In the low Atwood number limit:
\begin{equation}
	\psi = 1 + \frac{k \delta}{\sqrt{\pi}} .
\end{equation}

At larger amplitudes, the perturbations grow non-linearly with the light fluid rising through the heavier fluid in `bubbles' and the heavy fluid falling through the lighter fluid in `spikes.'
Early experiments by Davies and Taylor~\cite{Davies1950a} and potential flow models by Layzer~\cite{Layzer1955} for $A \approx 1$ suggested that the bubbles reach a terminal velocity, and later experiments by Dimonte and Schneider~\cite{Dimonte1996}, also at $A \approx 1$, showed that dense spikes free-fall.
On the other hand, recent experiments by Wilkinson and Jacobs~\cite{Wilkinson2007} and simulations by Ramaprabhu et al.~\cite{Ramaprabhu2006,Ramaprabhu2012}, Wei and Livescu~\cite{Wei2012}, and others~\cite{Sohn2011} show that, at Atwood numbers less than one half, the constant velocity regime is followed by a re-acceleration regime in which the velocity doubles.
The dynamics beyond re-acceleration have not been established, with Ramaprabhu et al. observing a return to the velocity of potential flow~\cite{Goncharov2002} while Wei and Livescu report continued constant acceleration.

Here, we consider the low-Atwood number limit.
The Boussinesq approximation, which ignores density differences that don't multiply the gravitational acceleration, simplifies the governing equations to a single incompressible phase with an active scalar representing the buoyancy:
\begin{align} \elabel{boussinesq}
\frac{D}{D t} u &= - \nabla P + \nu \nabla^2 u - A \vec{g} \phi, \\
\frac{D}{Dt} \phi &= D \nabla^2 \phi, \nonumber
\end{align}
where $u$ is the velocity,
$P$ is the pressure, and 
$\phi$ is the scalar that controls the density gradient.
Without loss of generality, we define $-1 \le \phi \le 1$.
These equations have a symmetry under inversion of the scalar and acceleration, $\phi \rightarrow -\phi, \hat{g} \rightarrow -\hat{g}$, so we know the bubbles and spikes have the same dynamics given the same initial conditions.
The parameter space of the equations are described by two non-dimensional numbers: the Grashof number,
\begin{equation} \elabel{grashof}
\text{Gr} = \frac{A g \lambda^3}{\nu^2},
\end{equation}
where $\lambda$ is a characteristic length, and
the Schmidt number,
\begin{equation} \elabel{schmidt}
\text{Sc} = \frac{\nu}{D}.
\end{equation}

We approximate the governing equations numerically using the spectral element method (SEM)~\cite{Deville2002}.
The SEM converges exponentially with respect to spectral order and has purely dispersive errors, making it a natural method for direct numerical simulations of mixing problems.
Unlike pseudo-spectral methods, it handles no-slip boundaries and can evenly sample the interior of the domain.
We use a specialized version of the Nek5000 community code, NekBox~\cite{NekBox2}, customized specifically to study the low Atwood number Rayleigh-Taylor instability.
NekBox restricts the domain to a tensor product of orthogonal bases and employs fast spectral coarse preconditioners for the pressure Poisson equation.
It is roughly an order of magnitude faster than the more general Nek5000.

It is the goal of this study to validate direct numerical simulations of the low-Atwood single mode Rayleigh-Taylor instability in NekBox against the best available experimental data, that from Wilkinson and Jacobs~\cite{Wilkinson2007}.
In the future, we will apply the same numerical methods to the broader question of the late-time dynamics of the low Atwood smRTI.

