\chapter{Related work}

The shortcomings of models for the single mode low Atwood Rayleigh-Taylor instability, specifically re-acceleration, were first recognized in 2006 by Ramaprabhu et al.~\cite{Ramaprabhu2006} based on numerical simulations.
Experimental confirmation by Wilkinson and Jacobs followed a year later~\cite{Wilkinson2007}.
Since then, multiple attempts have been made to capture re-acceleration in the models.
In this chapter, we review those attempts.

\section{Vortex ring correction of Ramaprabhu}

Ramaprabhu \etal ~\cite{Ramaprabhu2012} attribute the reacceleration to the formation of a vortex ring at the bubble tip.
They add a term to their buoyancy-drag model representing the centrifugal force per unit volume:
\begin{equation}
\left(\rho_2 g - \rho_1 g\right) + \rho_1 \frac{\omega_0^2 R}{ 4} = \frac{C_d \rho_2 v^2}{\lambda}
\end{equation}
where $\omega_0$ is the average vorticity in the bubble tip.
The model does not provide an evolution equation for $\omega_0$; it is measured from simulations ad-hoc making the model descriptive but not predictive.
The model agrees qualitatively, but not quantiatively, from the onset of stagnation at bubble height $h / \lambda \approx 0.5$ through re-acceleration at $h/\lambda \approx 2.0$, but doesn't capture linear growth at early times or the dynamics at late times.
Furthermore, they compare the vortex ring model to simulations using two different codes and the two codes disagree quantitatively over the re-acceleration regime and qualitatively over what follows it.

\section{Vorticity and viscosity in potential flow}

Banerjee \etal attempt to describe re-acceleration by adding viscous and vortical effects to a potential flow model~\cite{Banerjee2011}.
Similar to the vortex ring correction to buoyancy-drag, the vorticity is an input to the potential flow model.
Instead of using data from simulations, Banerjee \etal write the vorticity as an analytic function of time indepedent of the Atwood number.
The resulting dynamics have a single re-acceleration phase before reaching an asyptotic terminal velocity.

