\chapter{Related work}

The shortcomings of models for the single mode low Atwood Rayleigh-Taylor instability, specifically re-acceleration, were first recognized in 2006 by Ramaprabhu et al.~\cite{Ramaprabhu2006} based on numerical simulations.
Experimental confirmation by Wilkinson and Jacobs followed a year later~\cite{Wilkinson2007}.
Since then, multiple attempts have been made to capture re-acceleration in the models.
In this chapter, we review those attempts.

\section{Vortex ring correction of Ramaprabhu}

Ramaprabhu et al. attribute the reacceleration to the formation of a vortex ring at the bubble tip.
They add a term to their buoyancy-drag model representing the centrifugal force per unit volume:
\begin{equation}
\left(\rho_2 g - \rho_1 g\right) + \rho_1 \frac{\omega_0^2 R}{ 4} = \frac{C_d \rho_2 v^2}{\lambda}
\end{equation}
where $\omega_0$ is the average vorticity in the bubble tip.
The model does not provide an evolution equation for $\omega_0$; it is measured from simulations ad-hoc making the model descriptive but not predictive.

