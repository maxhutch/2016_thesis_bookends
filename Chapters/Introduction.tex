\chapter{Introduction}

\section{Formal definition}
The Rayleigh-Taylor instability occurs when the pressure and density gradients are in opposition:
\begin{equation}
(\nabla P)(\nabla \rho) < 0
\end{equation}
The canonical example of the Rayleigh-Taylor instability is a heavy fluid superposed over a lighter one in a gravitational field.
The standard terminology is based on this case.

Consider a horizontal planar interface at $z=0$ between two fluids of densities $\rho_h > \rho_l$, corresponding to the density of the heavier and lighter fluid, respectively.
The denser fluid is at $z > 0$ and the lighter at $z < 0$, and a gravitational acceleration $-g \hat{z}$.
In equilibrium, the pressure must balance the gravitational force:
\begin{equation}
P(x,y,z) = \begin{cases}- \rho_h g z + C& \text{ if } z > 0 \\
                        - \rho_l g z + C& \text{ else}
           \end{cases}
\end{equation}
A small perturbation is introduced, moving some heavy fluid below the interface and some light fluid above it but preserving the horizontal stratification of the pressure.
The forcing on the heavier fluid below the interface is then:
\begin{equation}\elabel{force1}
\sum F = - \nabla P + F_g = \rho_l g  - \rho_h g < 0,
\end{equation}
so the heavier fluid is forced downwards through the lighter fluid.
Conversely, the forcing on the lighter fluid above the interface is:
\begin{equation} \elabel{force2}
\sum F = - \nabla P + F_g = \rho_l g  - \rho_h g > 0,
\end{equation}
so the lighter fluid is forced upwards through the heavier fluid.
Perturbations in the interface grow, so the configuration is unstable.

\section{Instances and motivation}
The Rayleigh-Taylor instability is present in both natural and constructed systems at many scales.
This section describes a few such systems of particular importance.

\paragraph{Type Ia Supernovae}
In type Ia supernovae (SNe Ia), a white dwarf spontaneously ignites as its mass crosses the Chandrasekhar mass due to accretion from another source.
The fusion flame originates near the center of the star and burns outward.
The hot ash trailing the flame is lighter than the fuel, creating a Rayleigh-Taylor unstable flame front.
The primary RT instability and secondary Kelvin-Helmholtz instabilities wrinkle the flame front, enhancing mixing, burn rate, and flame speed~\cite{Zingale2005}.
These properties affect the rate of energy release and ejecta velocity, which can be observed.
The use of SNe Ia as standard candles underlines interest in their description.

\paragraph{Salt fingers and the thermohaline staircase}
Salt fingers are an instance of the Rayleigh-Taylor instability set up by a difference in the diffusivity of two mass carriers~\cite{Stern1969, Linden1973}.
Consider a fluid in which the density perturbation is a linear combination of two fields.
Let one of the fields have a stabilizing gradient and the other a destabilizing one.
If the destabilizing field is less diffusive than the stabilizing one, the system is unstable.
A parcel of fluid perturbed from its equilibrium height will equilibrate with the stabilizing field before the destabilizing one, resulting in a buoyant force that pushes the parcel further from vertical equilibrium.

In the oceanic case, the stabilizing field is temperature and the destabilizing field is salinity.
Near the surface, evaporation perturbs the salinity field creating parcels of salty dense fluid.
As they sink, the parcels cool more quickly than they diffuse salt, further increasing their density.
This flow drives tall vertical convective cells, called salt fingers, that mixes the oceans.

The same doubly diffusive buoyant process drives short broad convective cells at greater depths.
Observed as thermohaline staircase~\cite{Tait1971}, a series of sharp steps in the salinity and temperature vs depth, these cells extend occur at depths greater than a kilometer and extend for thousands of square miles.

\paragraph{ICF}
Inertial confinement fusion (ICF) is a fusion technique that confines hot dense plasma temporarily via an implosion, rather than magnetic field lines.
In ICF, the cryogenic hydrogen is coated with a plastic ablator forming very small hollow spheres, or microcapsules.
The microcapsules are dropped through a cylindrical hohlraum.
When the capsule enters the hohlraum, the hohlraum interior is illuminated with a high energy burst of laser light and radiates x-rays.
The x-ray radiation is absorbed by the plastic ablator causing it to rapidly expand and blow off the microcapsule, creating an implosion in the hydrogen fuel.

The implosion isn't perfectly uniform; the x-ray radiation provides non-uniform acceleration and there are asymmetries in the microcapsule.
These perturbations are Rayleigh-Taylor unstable: the dense plastic ablator is being accelerated through lighter hydrogen fuel.
The carbon in the ablator is much better at radiating energy than the hydrogen, so Rayleigh-Taylor mixing cool the fuel, preventing ignition.
~\cite{Goncharov}.

\section{Terminology}

In this section, we introduce common Rayleigh-Taylor terminology that will be used throughout the thesis.

\paragraph{Atwood number}
The Atwood number characterizes the density contrast:
\begin{equation} \elabel{atwood}
A = \frac{\rho_1 - \rho_2}{\rho_1 + \rho_2} \in (-1,1)
\end{equation}
where $\rho_1 > \rho_2$ corresponds to positive Atwood number.
Negative Atwood numbers result in stable oscillating interfaces.
There are three distinct regimes of unstable Atwood numbers: high, low, and moderate.

At high Atwood number, e.g. air and water, the flow is approximated in the limit of unit Atwood number.
The internal dynamics of the light fluid are decoupled from the heavy fluid, and the transfer of momentum through the interface into the dense fluid can be neglected.
In this regime, the flow of the dense fluid is nearly irrotational and potential flow models are reasonably accurate.
% When is it high enough?

At low Atwood number, e.g. salt water and fresh water, the flow is approximated in the limit of zero Atwood number.
The governing equations can be linearised, leading to the Boussinesq approximation: only terms with the product of the Atwood number and acceleration remain.
The behavior of the bubbles and spikes are symmetric given symmetric initial conditions.
It is often possible to treat the true two-phase problem as a single phase with an active scalar.
% When is it Boussinsq?

At moderate Atwood number, e.g. oil and water, not only is enough vorticity is generated to undermine potential flow models but also the density difference breaks the Boussinesq approximation and bubble-spike symmetry.
These are truly two-phase problems in that the fluids are strongly coupled and have different governing parameters, e.g. viscosity.

\paragraph{Boussinesq approximation}
When the density contrast, $\Delta \rho$, is small compared to the average density $\bar{\rho}$, then 
the governing equations can be linearized with respect to density.
This is known as the Boussinesq approximation and has the primary effect of neglecting differences in the inertia of the two fluids.

\paragraph{Uniform constant property}
In the spirit of the Boussinesq approximation, which neglects differences in the inertial of the two fluids, one can further assume the two fluids share all other material properties, such as viscosity.
When the flow is both Boussinesq and has uniform constant properties, the density can be modeled as an active scalar with a buoyant forcing term.

\paragraph{Miscible interface}
In most cases where the Boussinesq and uniform constant property approximations are valid, the interface between the two fluids is miscible.
For example, low-concentration solutions in a common solvent have interfaces governed by a diffusion coefficient, $D$, and small temperature gradients are governed by a thermal diffusivity $\alpha$.
In the limit where $D$ or $\alpha$ goes to zero, the interface is immiscible but there is no surface tension.
If there is no surface tension, the active scalar is governed by an advection-diffusion equation.

\paragraph{Incompressible}
The problem is said to be incompressible when each of the two fluids are.
Formally, this means the flow is divergence free, $\nabla \cdot u = 0$ and occurs when the Mach number is small.
In simulations, incompressibility has the effect of integrating out acoustic waves, significantly reducing the need to resolve small time-scales in the flow.

\paragraph{Single-mode}
The single mode Rayleigh-Taylor instability constrains the initial perturbation of the interface to a single pure frequency.
In 3D, this typically implies two orthogonal wavevectors with the same wavelength.
For the miscible uniform constant property Boussinesq Rayleigh-Taylor instability, the single mode initial condition is:
\begin{equation}
\phi(x,y,z,t=0) = A ~ \text{erf}\left(\frac{z + a_0 \cos(2 \pi x / \lambda) \cos(2 \pi y/\lambda)}{\delta}\right),
\end{equation}
where $a_0$ is the perturbation height,
$\lambda$ is the wavelength, and
$\delta$ is the interface thickness.
Typically, the perturbation height and interface thickness are chosen to initially be much smaller than the wavelength, $a_0, \delta << \lambda$.

\paragraph{Multi-mode}
The multi-mode Rayleigh-Taylor instability generically refers to the presence of more than one, and often many, wavelengths in the initial condition.
Experimentally, the multi-mode initial condition is usually `natural`, in the sense that it is not deliberately perturbed and instead is due to natural noise sources.
Computationally, the multi-mode condition is written as a sum of single modes with spectra $z_0(k) \sim |k|^{-p}$ for $p = 2$, $3/2$, or other rational factors.
These modes are randomized within the spectral envelop.

\paragraph{Governing equations}
Given a miscible interface between two Boussinesq, uniform constant property incompressible fluids, the governing equations can be recast in terms of an active scalar:
\begin{align}
\der{u}{t} + u \cdot \nabla u &= \nu \nabla^2 u - \nabla P - A g \phi \hat{z}\\
\der{\phi}{t} + u \cdot \nabla \phi &= D \nabla^2 \phi \\
\nabla \cdot u  &= 0,
\end{align}
where $u$ is the fluid velocity, 
$\phi$ is the active scalar that represents small mass differences,
and the acceleration is aligned vertically with $\hat{z}$.

These equations have three governing parameters; $(Ag)$, $\nu$, and $D$; which admit the construction of two dimensionless numbers:
\begin{equation}
\text{Grashof} = \frac{Ag L^3}{\nu^2}
\end{equation}
\begin{equation}
\text{Schmidt} = \frac{\nu}{D}
\end{equation}
If the initial and boundary conditions can be characterized by a single length scale, then these two parameters uniquely identify the system.
In the single-mode case, this length is the wavelength $\lambda$.

It is convenient to introduce another dimensionless number, the Rayleigh number:
\begin{equation}
\text{Rayleigh} = \text{Grashof} \cdot \text{Schmidt} = \frac{A g L^3}{\nu D},
\end{equation}
with which many mixing related quantities scale.
The three dimensionless numbers are abbreviated Gr, Sc, and Ra, respectively.

It is difficult to define a Reynolds number in the absense of a velocity parameter.
The square root of the Grashof number, which has the same scaling with the viscosity, is used instead.
It is sometimes called the perturbation Reynolds number, $\text{Re}_p = \sqrt{\text{Gr}}$.

\paragraph{Bubble height and spike depth}

Experiments and theories focus on the observation and explanation of a set of observables that are much smaller than the full degrees of freedom of the system.
The selection of these observables is informed by practical application and methodological analogy.

The bubble height refers to the distance beyond the initial interface that the bubble front has traveled, as a function of time.
The standard experimental definition projects the density onto a line normal to the initial interface and defines the front interface as the point at which the density is at its 99th or 95th percentile.
More formally:
\begin{equation}
H_p[\epsilon] = \sup \left\{z : \int \tilde\rho(x,y,z) dx dy < (1-\epsilon) \int \tilde\rho(x,y,\infty) dx dy \right\},
\end{equation}
where $\tilde\rho$ is the deviation from the mean density, $\tilde\rho = \rho - \bar\rho$.

If the fluids are miscible, this definition depends on the rate of diffusion across the interface.
To avoid dependence on diffusion across the interface, we can base the bubble height on a measurement of the equi-molar interface, which is stationary under diffusion.
To pick out the interface, we take a span-wise maximum instead of a span-wise sum:
\begin{equation}
H_m[\epsilon] = \sup \left\{z : \max_{x,y} \tilde\rho(x,y,z) < 0 \right\}
\end{equation}

However, diffusive mixing across the sides of an elongated bubble dilute as it grows.
This dilution, which is observed as a linear profile in the span-wise maximum instead of an error function profile, can dip below $\tilde\rho = 0$, at which point the growth of $H_m$ is also influenced by mixing.

In the absence of this affect, $H_{m}$ tracks not only the equi-molar surface but also the inflection point in the profile $\max_{x,y} \tilde \rho$.
This inflection point closely tracks the equi-molar surface at low diffusivity but is robust to diffusion across the bubble, remaining in the center of the error function region of the maximum density profile.
Formally, this definition is:
\begin{equation}
H_i[\epsilon] = \sup \left\{z : \frac{d^2}{dz^2} \max_{x,y} \tilde\rho(x,y,z) = 0 \right\}
\end{equation}

Equivalent definitions can be given for the spike depth.
If the flow is Boussinesq and the initial condition is symmetric, the bubble height and spike depth can be averaged.

\paragraph{Mixing width}

In some application, the interpenetration of the two fluids is secondary to their mixing.
To measure the mixed-ness of the flow, we map the pure fluids to unity, mixed fluids to zero, and integrate.
First, we define a normalized scalar measure of the density as:
\begin{equation}
\phi = \frac{2\rho - 2\bar\rho}{\rho_1 - \rho_2} \in \left[-1,1\right],
\end{equation}
then the purity of a fluid volume is $|\phi|$.
The mix volume is simply the integral:
\begin{equation}
\Theta(t) = \int \left(1 - \left|\phi(x,y,z,t)\right|\right) dV,
\end{equation}
and the mixing width $\theta = \Theta / A$, where $A$ is the span-wise extent of the volume integral.

