\chapter{Introduction}

\section{Formal definition}
The Rayleigh-Taylor instability occurs when the pressure and density gradients are in opposition:
\begin{equation}
(\nabla P)(\nabla \rho) < 0
\end{equation}
The canonical example of the Rayleigh-Taylor instability is a heavy fluid superposed over a lighter one in a gravitational field.
The standard terminology is based on this case.

Consider a horizontal planar interface at $z=0$ between a fluids of densities $\rho_h > \rho_l$ with the denser fluid at $z > 0$ and the lighter at $z < 0$, and a gravitational acceleration $-g \hat{z}$.
In equilibrium, the pressure must balance the gravitational force:
\begin{equation}
P(x,y,z) = \begin{cases}- \rho_h g z + C& \text{ if } z > 0 \\
                        - \rho_l g z + C& \text{ else}
           \end{cases}
\end{equation}
A small perturbation is introduced, moving some heavy fluid below the interface and some light fluid above it.
The forcing on the heavier fluid below the interface is:
\begin{equation}\elabel{force1}
\sum F = - \nabla P + F_g = \rho_l g  - \rho_h g < 0
\end{equation}
so the heavier fluid is forced downwards through the lighter fluid.
Conversely, the forcing on the lighter fluid above the interface is:
\begin{equation} \elabel{force2}
\sum F = - \nabla P + F_g = \rho_l g  - \rho_h g > 0
\end{equation}
so the lighter fluid is forced upwards through the heavier fluid.
Perturbations in the interface grow, so the configuration is unstable.

\section{Instances and motivation}
The Rayleigh-Taylor instability is present in both natural and constructed systems at many scales.
This section describes a few such systems of particular importance.

\paragraph{Type Ia Supernovae}
One of the earliest motivations for the study of the Rayleigh-Taylor instability comes supernovae. 

\paragraph{Salt fingers}
Salt fingers are an instance of the Rayleigh-Taylor instability set up by a difference in the diffusivity of two mass carriers, in this case salinity and temperature.

\paragraph{ICF}
Inertial confinement fusion (ICF) is a fusion technique that confines hot dense plasma temporarily via an implosion, rather than magnetic field lines.
In ICF, the cryogenic hydrogen is coated with a plastic ablator forming very small hollow spheres, or microcapsules.
The microcapsules are through a cylindrical hohlraum.
When the capsule is at the center of the hohlraum, the hohlraum is illuminated with a high energy burst of laser light and radiates x-rays.
The x-ray radiation is absorbed by the plastic ablator causing it to rapidly expand and blow off the microcapsule, creating an implosion in the hydrogen fuel.

The implosion isn't perfectly uniform; the x-ray radition provides non-uniform acceleration and there are asymmetries in the microcapsue.
These perturbations are Rayleigh-Taylor unstable: the dense plastic ablator is being accelerated through lighter hydrogen fuel.
The carbon in the ablator is much better at radiating energy than the hydrogen, so as the RTI mixes the plastic into the fuel the fuel cools, preventing ignition.

\paragraph{Reactors}


\section{Terminology}

\paragraph{Atwood number}
The Atwood number characterizes the density contrast:
\begin{equation} \elabel{atwood}
A = \frac{\rho_1 - \rho_2}{\rho_1 + \rho_2} \in (-1,1)
\end{equation}
where $\rho_1 > \rho_2$ corresponds to positive Atwood number.
There are three distinct regimes for the Atwood number: high, low, and moderate.

At high Atwood number, e.g. air and water, the internal dynamics of the light fluid are decoupled from the heavy fluid.
The transfer of momentum through the interface into the dense fluid can be neglected.
In this regime, potential flow models are reasonable.

At low Atwood number, e.g. salt water and fresh water, the governing equations can be linearized about the Atwood number, leading to the Boussinesq approximation.
The behavior of the bubbles and spikes are symmetric given symmetric initial conditions.

At moderate Atwood number, e.g. oil and water, not only is enough vorticity is generated to undermine potential flow models but also the density difference breaks bubble-spike symmetry.

\paragraph{Boussinesq approximation}
When the density contrast, $\Delta \rho$ is small compared to the average density $\bar{\rho}$, then 
the governing equations can be linearized with respect to density.
This is known as the Boussinesq approximation and has the primary effect of neglecting differences in the inertia of the two fluids.

\paragraph{Constant property}
In the spirit of the Boussinesq approximation, which neglects differences in the inertial of the two fluids, one can further assume the two fluids share all other material properties, such as viscosity.
When the flow is both Boussinesq and has constant properties, the density can be modeled as an active scalar with a buoyant forcing term.

\paragraph{Miscible interface}
In most cases where the Boussinesq and constant property approximations are valid, the interface between the two fluids is miscible.
For example, low-concetration solutions in a common solvent have interfaces governed by a diffusion coefficient, $D$, and small temperature graidents are governed by a thermal diffusivity $\alpha$.
In the limit where $D$ or $\alpha$ goes to zero, the interface is immiscible but there is no surface tension.

\paragraph{Single-mode}
The single mode Rayleigh-Taylor instability constraints the initial perturbation of the interface to a single pure frequency.
In 3D, this typically implies two orthogonal wavevectors with the same wavelength.
For the miscible constant property Boussinesq Rayleigh-Taylor instability, the single mode initial condition is:
\begin{equation}
\phi(x,y,z,t=0) = A \text{erf}\left(\frac{z + a_0 \cos(2 \pi x / \lambda) \cos(2 \pi y/\lambda)}{\delta}\right),
\end{equation}
where $a_0$ is the perturbation height,
$\lambda$ is the wavelength, and
$\delta$ is the interface thickness.
Typically, the perturbation height and interface thickness are chosen to initially be much smaller than the wavelength, $a_0, \delta << \lambda$.

\paragraph{Multi-mode}
The multi-mode Rayleigh-Taylor instability generically refers to the presense of more than one, and often many, wavelengths in the initial condition.
Experimentally, the multi-mode initial condition is usually `natural`, in the sense that it is not deliberately perturbed and instead is due to natural noise sources.
Computationally, the multi-mode condition is written as a sum of single modes with spectra $a_0(k) \sim |k|^{-p}$ for $p = 2, 3/2, \ldots$.

\paragraph{Governing equations}
Given a miscible interface between two Boussinesq,  constant property fluids, the governing equations can be recaste in terms of an active scalar:
\begin{align}
\der{u}{t} + u \cdot \nabla u &= \nu \nabla^2 u - \nabla P - A g \phi \hat{z}\\
\der{\phi}{t} + u \cdot \nabla \phi &= D \nabla^2 \phi \\
\nabla \cdot u  &= 0
\end{align}
These equations have three governing parameters; $Ag$, $\nu$, and $D$; which admit the construction of two dimensionless numbers:
\begin{equation}
\text{Grashof} = \frac{Ag L^3}{\nu^2}
\end{equation}
\begin{equation}
\text{Schmidt} = \frac{\nu}{D}
\end{equation}
If the initial and boundary conditions can be characterized by a single length scale, then these two parameters uniquely identify the system.

\paragraph{Bubble height and spike depth}

Experiments and theories focus on the observation and explanation of a set of observables that are much smaller than the full degrees of freedom of the system.
The selection of these observables is informed by practical aplication and methodological analogy.

The bubble height refers to the distance beyond the initial interface that the bubble front has traveled, as a function of time.
The standard experimental definition projects the density onto a line normal to the initial interface and defines the front interface as the point at which the density is at its 99th or 95th percentile.
More formally:
\begin{equation}
H_p[\epsilon] = \sup \left\{z : \int \tilde\rho(x,y,z) dx dy < (1-\epsilon) \int \tilde\rho(x,y,\infty) dx dy \right\},
\end{equation}
where $\tilde\rho$ is the deviation from the mean density, $\tilde\rho = \rho - \bar\rho$.

If the fluids are miscible, this definition depends on the rate of diffusion across the interface.
To avoid dependence on diffusion across the interface, we can base the bubble height on a measurement of the equimolar interface, which is stationary under diffusion.
To pick out the interface, we take a span-wise maximum instead of a span-wise sum:
\begin{equation}
H_m[\epsilon] = \sup \left\{z : \max_{x,y} \tilde\rho(x,y,z) < 0 \right\}
\end{equation}

However, diffusive mixing across the sides of an elongated bubble dillute as it grows.
This dillusion, which is observed as a linear profile in the span-wise maximum instead of an error function profile, can dip below $\tilde\rho = 0$, at which point the growth of $H_m$ is also influenced by mixing.
In the absense of this affect, $H_{m}$ tracks not only the equimolar surface but also the inflection point in the profile $\max_{x,y} \tilde \rho$.
This inflection point closly tracks the equimolar surface at low diffusivities but is robust to diffusion across the bubble, remaining in the center of the error function region of the maximum density profile.
Formally, this definition is:
\begin{equation}
H_m[\epsilon] = \sup \left\{z : \frac{d^2}{dz^2} \max_{x,y} \tilde\rho(x,y,z) = 0 \right\}
\end{equation}

Equivalent definitions can be given for the spike depth.
If the flow is Boussinesq and the initial condition is symmetric, the bubble height and spike depth can be averaged.

\paragraph{Mixing width}



\begin{comment}
\section{Stages}

\subsection{Linear growth} 
When the amplitude of the surface perturbation is small compared to its wave-length, the perturbation grows exponentially.
The growth rate is seen to depend on the forcing, wave-length, viscosity, diffusivity, and interface thickness.
This stage can be treated with linear and weakly-nonlinear theories, as discussed further in \sref{linear}.


\subsection{Single-mode stagnation}
As the growing Rayleigh-Taylor modes saturate, the acceleration of the bubble tip decreases.
The stagnation stage of the single-mode Rayleigh-Taylor instability is defined as a stage of the flow that exhibits nearly constant bubble velocity.
It had been believed that the first such stage was terminal, i.e. steady state.
It has recently been shown, via both simulation~\cite{Ramaprabhu2006} and experiment~\cite{Wilkinson2007}, that at low Atwood number and high Reynolds number the flow continues to develop into re-acceleration stage.
For this reason, the nearly constant velocity is termed the \textit{stagnation velocity}.
The primary observable of a stagnation stage is the stagnation velocity, but there is secondary interest in the velocity profile through the transition from the weakly non-linear to stagnation stages and in the bubble height at which stagnation velocity is reached.

\subsection{Single-mode late time}
For low Atwood and high Reynolds numbers, the stagnation stage is followed by a reacceleration stage around unity aspect ratio.
Reacceleration has been seen in both numerical simulations \cite{Ramaprabhu2006, Ramaprabhu2012, Wei2012} and experiments{Wilkinson2007}.
Following the immediate reacceleration, there is disagreement as to whether or not the single mode problem ultimatly reaches a terminal stage.
Ramaprabhu \etal suggest that the re-acceleration is transient, with the bubble velocity ultimately returning to a potential-flow-like value.
Wei and Livescu, on the other hand, suggest the terminal stage to be one of `chaotic development` with quadratic growth dynamics.
Unfortunately, the late-time stages have yet to be accessed experimentally.

\subsection{Multi-mode}
Under natural multi-mode initial conditions, the linear growth modes couple as the aspect ratio increases, forming bubble and spike structures.
The bubbles and spikes interact with one another, competetively and constructively, leading to successively larger structures.
It is generally observed that the multi-mode aspect ratio, $D_b / h$, remains nearly constant.
\end{comment}
